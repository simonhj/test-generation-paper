% !TEX root = paper.tex
\section{Related work}

At first blush, this problem seems to be reminiscent of the problem of generating tests for programs 
written as control-flow graphs, with the goal of exercising each acyclic path in the program, if
possible.  There have been a number of techniques in the literature for test generation. Mostly
notably, techniques such as \textit{concolic} testing attempt to identify a series of test data 
that would force program execution thorough different paths.  Other approaches are based on model
checking, with the goal of creating test inputs to reach specific program states.

However, the problem of test generation from business rules is different.  Business rules do not
describe the implementation of a system: rather they only describe a model.  (Model-based test 
generation?)

\subsection{AI Planning}

Previous work has applied AI planning to software testing~\cite{Scheetz99ai,Howe97testcase}. The sequencing of operations done by our algorithm is similar to the AI planning problem(cite) where actions with pre- and postconditions are sequenced by a planner algorithm using either forward or backward chaining. The algorithms used by AI planners have been proven insufficient for the size of models in our domain (backref). 

Part of an AI planning problem is the initial state of the program, including what objects exists and which conditions that can be assumed. Our technique does not need this: The types of objects that exist in the system are defined in the model along with operations that create them. A test sequence will construct any object needed to cover a given rule. 

Paradkar et al~\cite{conf/icws/ParadkarSWJOSL07}.\ presents a system for testing web services specified in the semantic markup language OWL-S. OWL-S is similar to our modeling language but does not have any well-formedness properties. The focus of the presented technique is conformance testing, that is, testing if a given implementation conforms to the specification whereas our approach aims to cover a set of business rules with tests. The paper does not mention the size of the models the technique was evaluated on or the size of the generated test sequences.


\subsection{Model based testing}

...