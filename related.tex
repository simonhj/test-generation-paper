% !TEX root = paper.tex
\section{Related Work}

At first blush, this problem seems to be reminiscent of the problem of
generating tests for programs written as control-flow graphs, with the
goal of exercising each acyclic path in the program, if possible.
There have been a number of techniques in the literature for test
generation. Mostly notably, techniques such as \textit{concolic}
testing attempt to identify a series of test data that would force
program execution thorough different paths~\cite{dart, concolic}.  Other
approaches are based on model checking, with the goal of creating test
inputs to reach specific program states.

However, the problem of test generation from business rules is
different. Business rules do not describe the implementation of a
system: rather they only describe a model and many of the concerns
that arise when dealing with control flow graph derived from real code
are not pertinent.

\subsection{Model Based Testing}

The set of business rules can viewed as a model of an actual system
that supports them and our approach then becomes a variant of model
based test generation~\cite{utting2012}.

Typical modeling languages used by model based testing systems include
UML sequence diagrams~\cite{nayak2009}, modeling specific languages
such as the Systems Modeling Language~\cite{friedenthal2011} and
finite state machine notations such as UML state
charts~\cite{offhut99}. While the business rules as presented in this
paper could be expressed in any of these notations it would be
cumbersome and require some degree of non-trivial encoding on part of
the author, making it unattractive to a non-programmer. Our business
rule language is designed so that rules are written in much the same
way as one would write them in prose thereby making it more accessible
to non-programmers.

Sriganesh and Ramanathan~\cite{sriganesh2012} extract business rules
from systems described in the Business Process Model and Notation
language, a graphical notation used to describe business
processes. The rule output is in a format similar to our formalism and
our algorithm could conceivably be applied to generate test sequences
from these rules.

\subsection{AI Planning}

Previous work has applied AI planning to software
testing~\cite{Scheetz99ai,Howe97testcase}. The sequencing of
operations done by our algorithm is similar to the AI planning
problem~\cite{Weld94} where actions with pre- and postconditions are sequenced
by a planner algorithm using either forward or backward chaining. The
algorithms used by AI planners are similar the naive technique
presented in Section~\ref{sec:eval} which proved insufficient for our
benchmarks. 

Part of an AI planning problem is the initial state of the program,
including what objects exists and which conditions that can be
assumed. Our technique does not need this: The types of objects that
exist in the system are defined in the model along with operations
that create them. A test sequence will construct any object needed to
cover a given rule part.

Paradkar et al~\cite{conf/icws/ParadkarSWJOSL07}.\ presents a system
for testing web services specified in the semantic markup language
OWL-S. The system is backed by an AI planner. OWL-S represents
operations in a manner similar to our language with pre- and
postcondtions. However each operation has only one associated pre- and
postcondition pair making it more akin to a rule in our language. If
one where to translate a model in our business rule language into
OWL-S each rule would be translated into a standalone operation. Such
a translation could potentially increase the size of the search space
and lead to spurious sequences being constructed. The focus of the
technique presented in the paper is on conformance testing, that is,
testing if a given implementation conforms to the specification. In
our approach the business rules are the specification and the aim is
instead to generates tests that cover all rules. The paper does not
mention the size of the models the technique was evaluated on nor the
size of the generated test sequences.

\subsection{Method Sequence Generation}

Do we need this section?

When testing object oriented code achieving high branch coverage of a
method requires driving the receiver object and any objects passed as
arguments into a desired state. One approach is to produce the desired
object state by generating sequences of method calls that create and
mutate objects similar to how we sequence
operations~\cite{pacheco2007,tillmann2008,thummalapenta2011}. However
the techniques are usually either based on random exploration or
symbolic execution neither of which apply in our business rule setting.

Something about unsat core
