\section{Related Work}
\label{sec:related}

At first blush, this problem seems to be reminiscent of the problem of
generating tests for programs written as control-flow graphs, with the
goal of exercising each acyclic path in the program, if possible.
There have been a number of techniques in the literature for test
generation. Mostly notably, techniques such as \textit{concolic}
testing attempt to identify a series of test data that would force
program execution thorough different paths~\cite{dart, concolic}.  Other
approaches are based on model checking, with the goal of creating test
inputs to reach specific program states.

However, the problem of test generation from business rules is
different. Business rules do not describe the implementation of a
system: rather they only describe a model and many of the concerns
that arise when dealing with control flow graph derived from real code
are not pertinent.

\subsection{Model Based Testing}

The set of business rules can viewed as a model of an actual system
that supports them and our approach then becomes a variant of model
based test generation~\cite{utting2012}.

Typical modeling languages used by model based testing systems include
UML sequence diagrams~\cite{nayak2009}, modeling specific languages
such as Systems Modeling Language~\cite{friedenthal2011} and
finite state machine notations such as UML state
charts~\cite{offhut99}. While the business rules as presented in this
paper could be expressed in any of these notations, it would be
cumbersome and require some degree of non-trivial encoding on part of
users, making it unattractive to a non-programmer such as a business analyst. Our business
rule language is designed to be user friendly, where rules are written 
in much the same way as one would write them in prose thereby making it more accessible
to non-programmers.

Sriganesh and Ramanathan~\cite{sriganesh2012} extracts business rules
from systems described in the Business Process Model and Notation
language, a graphical notation used to describe business
processes. The rule output is in a format similar to our formalism and
our algorithm could conceivably be applied to generate test sequences
from those rules.

The work on combinatorial test optimization~\cite{segall2012} also requires a model 
of an application as a starting point, but that model is typically not rich
enough to allow generation of test scenarios; the model does not have an operation-based
view of the application.   However, test optimization 
can be carried out in a post pass over the tests generated by our technique.

\subsection{AI Planning}

Previous work has applied AI planning to software
testing~\cite{Scheetz99ai,Howe97testcase}. The sequencing of
operations done by our algorithm is similar to the AI planning
problem~\cite{Weld94} where actions with pre- and postconditions are sequenced
by a planner algorithm using either forward or backward chaining. The
algorithms used by AI planners are similar to \exhaust{}
(presented in Section~\ref{sec:eval}) and proved to be insufficient for our
benchmarks. Part of an AI planning problem is the initial state of the program,
including what objects exist and what conditions can be
assumed. Our technique does not need such information---the types of objects that
exist in the system are defined in the model along with operations
that create them. 
%A test sequence will construct any object needed to
%cover a given rule part.

Paradkar et al~\cite{conf/icws/ParadkarSWJOSL07}.\ presents a system
for testing web services specified in the semantic markup language
OWL-S~\cite{owls}. The system is backed by an AI planner. OWL-S represents
operations in a manner similar to our language with pre- and
postcondtions. However each operation has only one associated pre- and
postcondition pair making it more akin to a rule in our language. If
one where to translate a model in our business rule language into
OWL-S, each rule would be translated into a standalone operation. Such
a translation could potentially increase the size of the search space
and lead to spurious sequences being constructed. The focus of the
technique presented in the paper is on conformance testing, that is,
to test whether a given implementation conforms to the specification. In
our technique, business rules are the specification and the aim is
instead to generate sequences that cover all rules. The paper does not
mention the size of the models the technique was evaluated on nor the
size of the generated test sequences.

\subsection{Testing Object-Oriented Programs}

Our work is also related to existing techniques that focus on test 
generation for object-oriented programs~\cite{pacheco2007,tillmann2008,thummalapenta2011, xie:symstra,
csallner:jcrasher, phil:evo, tonella:etoc}. At a high level,
these techniques also focus on inferring operation sequences 
with a goal of generating different object states for receiver
or arguments of method under test so as to achieve high code coverage.
Among these techniques, Symstra~\cite{xie:symstra} generates sequences
exhaustively up to a small bound on sequence length. 
Randoop~\cite{pacheco2007} and JCrasher~\cite{csallner:jcrasher} generate
sequences by randomly selecting methods that produce instances of receiver
or argument types. Tonella~\cite{tonella:etoc} and McMinn and Holcombe~\cite{phil:evo}
use evolutionary approaches that accept an initial set of sequences
and evolve those sequences based on a fitness function. Among all
these techniques, our work is closely related to Seeker~\cite{thummalapenta2011}
that also attempts to compose sequences incrementally based on branches that
are not yet covered. The major contribution of Seeker is to infer
candidate operations especially when those branches include private fields,
since private fields cannot be modified directly. In our current work,
we do not face that issue, since our model does not have the notion of public
and private attributes. Our current work extends Seeker by extracting
unsatisfied core and using to guide the search process. 
We believe that Seeker can greatly benefit with
our new algorithm of leveraging unsatisfied core as a guidance for suggesting
candidate operations.

%Finally, our previous work \wateg{}~\cite{Thummalapenta:2013} also accepts rules 
%and generates tests by crawling an application's GUI. However, the rules handled in our previous
%work are primarily related to application behaviors such as access control
%or navigational properties. Thus, those rules are expressed 
%using GUI elements such as links, text boxes, or buttons. In contrast,
%in our current work, our rules express business logic and the goal is to
%generate sufficient test data to exercise those rules. 
