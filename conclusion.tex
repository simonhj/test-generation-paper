\section{Conclusion}

In this paper, we presented a new domain-specific language for modeling business
rules that can capture functional specifications of enterprise
systems. We also defined well-formedness properties on the model that can be verified
mechanically. We implemented our tool as an Eclipse plug-in, where
non-programmers also can create and refine models in a guided fashion
We also presented a novel technique, based on unsatisfied cores, that generates test sequences
by translating rules in to logical expressions.

%We presented an algorithm that mechanically generates test sequences to
%exercise rules in the model. The algorithm translates the rules in the
%model into logical expressions and uses a constraint solver to infer
%the needed input values. For optimization, the algorithm uses a novel
%approach to prune the search space based on unsatisfiable cores. 

Our technique was evaluated using three models that were derived from business rules written
in English. The results show that our technique
is able to cover 99\% of all business rules. Toward our longer-term
vision of bringing end-to-end automation to testing of enterprise
applications, we will investigate the integration between \tool{} and
\wateg{} (our previous work) to generate executable test cases for validating an
application's conformance to business rules.

%% This work is part of our longer-term vision of generating executable
%% GUI scripts, where, for each operation, we plan to extend support for
%% users to provide flow specifications for operations. In our future
%% work, we plan to integrate the current technique with our previous
%% work, where \tool{} generates sequences (with necessary test data) and
%% then \wateg{} accepts sequences and flow-specifications for each
%% operation in those sequences as inputs, and automatically generates
%% executable test cases by crawling the application's GUI.
