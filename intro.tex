\section{Introduction}

A business rule articulates some aspect of the expected functional behavior (or a \textit{requirement})
of an enterprise application. Here is a simple business rule that determines how an invoice total is 
determined in a billing application:

\begin{quote}
	The \textit{Balance Type} of a customer affects how invoice total is computed; it can be 
	one of the following:

	\textit{None}: The customer's account will not hold a balance; instead all charges accrued 
	in an order will be included in the next invoice;
	
	\textit{Credit}: The customer's account may accrue charges up to the set credit limit. 
	Charges will automatically be paid from the users credit pool until the set limit is reached. 
	Users are responsible for paying their credit debt as well as any overages.
\end{quote}	

We will examine this rule closely later; for now, suffice it to say that the requirements 
of an enterprise system are typically captured by a large number (often in hundreds) of business rules
such as the one above.

It is reasonable to expect functional testing of an enterprise system to \textit{cover} its 
business rules, which is to say, testing would exercise every distinct scenario described in each of
its business rules.  For example, in the above rule, one of the scenario to be exercised is
that a customer's balance type is credit, and that the order amount does not exceed the customer's credit limit.
A test that exercises this scenario would set up a customer with the balance type as credit as well as a certain
credit limit, create an order with the appropriate constraint on the order total, and then finally
create an invoice for that customer, and verify the invoiced amount.  (See Figure~\ref{fig:jbilling-flow}.) 
Thus, it requires a carefully thought out test scenario (i.e. a sequence of test steps) as well as associated 
test data (i.e. values to be entered in the relevant test steps) to exercise a business rule, or a scenario 
thereof.

\begin{figure*}
\centering
\includegraphics[trim=0 200 0 50,clip,width=7.5in]{figs/jbilling-flow}
\caption{Test sequence for a business rule from jBilling application}
\label{fig:jbilling-flow}
\end{figure*}

In practice, due to time pressures, testers are more often ad hoc than methodical in creating test 
scenarios and test data.  One part of the problem is that a realistic system might have hundreds of 
business rules written out in plain text, and it is difficult to grasp a global view of how the rules 
together describe the application behavior.  A related part of the problem is that it may need complex 
reasoning to piece together test sequences that would cover each scenario. Consequently, testers end up 
creating a multitude of tests that exercise the same business rule, or a scenario thereof, over and over 
again without any additional benefit (especially if they are incentivised by the number of tests rather than 
quality), and more problematically, may neglect to create tests for some other business rules.  The net 
result is that despite a lot of resources spent in testing, bugs still escape into the field.

Our vision is to make testing of enterprise software more tool based, by adapting technology developed 
for automated and systematic test generation for programs.   In this vision, business rules would be 
written in a structured notation that allows mechanized analysis. 
(Special editors could be created to enable non-programmers to capture business rules in a structured notation; 
this is an independent challenge in \textit{end user} programming).  A tool would validate business rules 
and point out any ambiguities or omissions that it can detect.  After the business rules pass validation, 
another tool would generate test sequences and test data to exercise the application thoroughly as well as
without redundancies.

We have built a system to partially fulfil this vision.  In the rest of this introductory section, we
give an overview of our system, describe some of the challenges in automating test data generation, and 
briefly summarize our results.

%Randomly generated test data cannot be expected to suffice for enterprise
%applications with complex rules. Also, systematic test-generation approaches
%based on program analysis (\eg \cite{Emmi:2007,Li:2010,Marcozzi:2012,Pan:2011})
%cannot be expected to tackle enterprise applications, which use a mix of
%multiple language and database technologies in their implementation. Moreover,
%these techniques are directed toward attaining simple forms of code coverage,
%such as statement or branch coverage, rather than coverage of complex business
%rules.

\subsection{An overview}

Enterprise systems of interest to us are transaction oriented, which means that they consist of a set of
transactions or operations (e.g. create a customer, add an item to an order, and so on) that operate on databases.  
A business rule applies to a particular operation supported by the system.  Formally, a business rule
describes the relation between the database state before the operation and after it.  Figure~\ref{fig:invoice} shows
formalization of the business rule quoted informally at the beginning of the introduction.  It says that
the operation refers to an invoice record, \textit{inv}, and modifies specific attributes of \textit{inv}.
There are three scenarios that occur in this rule.  The first scenario applies when the customer to which the
invoice refers has balance type \textit{None}; the \textit{precondition} is shown on the left of the first arrow.  
The second scenario applies when the customer has balance type \textit{credit}, but the credit limit is sufficient 
to cover the order total.  (Note that the invoice has references to the customer to which this invoice pertains, 
as well as an order created in the system; in a relational database, these would be foreign keys in customer and 
order table.)  The third scenario applies when the customer has balance type \textit{credit}, but the order total 
exceeds the credit limit.  In each scenario, the effect of the operation is to compute invoice total, and update 
the customer's residual credit limit; this effect, or \textit{postcondition} is shown to the right of the arrow 
in each case.  Note that business rules refer to state observable at transaction boundaries; intermediate program 
states encountered in the implementation while a transaction is in process, are not important to business rules.

\begin{figure}
\centering
\includegraphics[trim=30 300 270 0,clip,width=\columnwidth]{figs/invoice}
\caption{Business rule for computing invoice amount}
\label{fig:invoice}
\end{figure}

To cover the three scenarios in business rules, it is necessary to create each of the three preconditions,
and in each case verify that the post condition holds after the operation has completed.  This brings us to the 
main difficulty in creating tests:  appropriate database state, such as a customer with a certain balance type,
and an order with a certain amount, needs to be established before any of these scenarios can be exercised.  
We showed in Figure~\ref{fig:jbilling-flow} the steps that would be required to create these preconditions 
(applicable to each of the three scenarios, though the test data would need to be different).   How can we
identify these steps, and the data to be entered in each of these steps automatically to cover a business rule, or
one of its scenarios?

Our observation is that the steps that are required to drive the database state to a desired precondition are
carried out by operations, and those operations too would have rules that specify their functionality.  We could
then use business rules as ``state transformers'' and piece together a sequence of operations to arrive at
a desired state.  The piecing together can be carried out in many ways, and in similar situations, planning
techniques from AI have been used.  

In this work, we draw inspiration from weakest preconditions.

\subsection{Challenges}

At first blush, this problem seems to be reminiscent of the problem of generating tests for programs 
written as control-flow graphs, with the goal of exercising each acyclic path in the program, if
possible.  There have been a number of techniques in the literature for test generation. Mostly
notable, techniques such as \textit{concolic} testing attempt to identify a series of test data 
that would force program execution thorough different paths.  Other approaches are based on model
checking, with the goal of creating test inputs to reach specific program states.

However, the problem of test generation from business rules is different.  Business rules do not
describe the implementation of a system: rather they only describe a model.  (Model-based test 
generation?)
