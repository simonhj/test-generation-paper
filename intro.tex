\section{Introduction}

A business rule articulates some aspect of the expected functional behavior (or
a \textit{requirement}) of an enterprise application. Here is a simple business
rule that determines how an invoice total is determined in a billing
application:
%
\begin{quote}
	The \textit{Balance Type} of a customer affects how invoice total is computed; it can be 
	one of the following:

	\textit{None}: The customer's account will not hold a balance; instead all charges accrued 
	in an order will be included in the next invoice;
	
	\textit{Credit}: The customer's account may accrue charges up to the set credit limit. 
	Charges will automatically be paid from the users credit pool until the set limit is reached. 
	Users are responsible for paying their credit debt as well as any overages.
\end{quote}	
%
We will examine this rule closely later; for now, suffice it to say that the
requirements of an enterprise system are typically captured by a large number
(often, hundreds) of business rules such as the one above.

It is reasonable to expect functional testing of an enterprise system to
\textit{cover} its business rules, which is to say, testing would exercise every
distinct scenario described in each of its business rules.  For example, in the
preceding rule, one of the scenario to be exercised is that a customer's balance
type is credit, and that the order amount exceeds the customer's credit limit.
A test that exercises this scenario would set up a customer with the balance
type as \textit{credit} as well as a certain credit limit, say, \textit{100},
create an order and add items to the order to bring a total to, say,
\textit{120}, to exceed the credit limit for that customer, and then finally
create an invoice for that order,and verify the invoiced amount.
Figure~\ref{fig:jbilling-flow} illustrates this flow.  Although the values
\textit{credit}, \textit{100}, and \textit{120} can be identified just from this
rule (by constraint solving), identifying a test sequence is also important to
apply those values at the right fields on the appropriate
screens.\footnote{\small Not all fields on the screens in
  Figure~\ref{fig:jbilling-flow} are constrained from the point of view of
  exercising a particular scenario, but the application might still demand
  sensible values for them. The tester is expected to make these up.}

\begin{figure*}
\centering
\includegraphics[trim=47 210 35 50,clip,width=\textwidth]{figs/jbilling-flow}
\vspace*{-10pt}
\caption{Test sequence for exercising a business rule from the
  \subject{jBilling} application.}
\vspace*{-7pt}
\label{fig:jbilling-flow}
\end{figure*}

It requires a carefully thought-out test scenario, \ie a sequence of test steps
as well as associated test data, \ie values to be entered in the relevant test
steps to exercise a business rule, or a scenario thereof.  Without systematic
test creation, testers may end up creating multiple tests that exercise the same
business rule, or a scenario thereof, over and over again without any additional
benefit (especially if they are incentivised by the number of tests created
rather than quality of the created tests), and more problematically, may neglect
to create tests for some other business rules.

In practice, due to time pressures, testers are more often ad-hoc than
methodical in creating test scenarios and test data.  One part of the problem is
that a realistic system might have hundreds of business rules written out in
plain text, and it is difficult to grasp a global view of how the rules together
describe the application behavior.  A related part of the problem is that it may
need complex reasoning to piece together test sequences that would cover each
scenario.  The net result is that, despite a lot of resources spent in testing,
bugs still escape into the field.

\hyphenation{non-prog-ram-mers}

Our vision is to make testing of enterprise software more tool-based, by
adapting technology developed for automated and systematic test generation for
programs.  In this vision, business rules would be written in a structured
notation that allows mechanized analysis.  Special editors could be created
to enable non-programm\-ers to capture business rules in a structured notation;
this is an independent challenge in \textit{end-user} programming.  A tool would
validate business rules and point out any ambiguities or omissions that it can
detect.  After the business rules pass validation, another tool would generate
test sequences and test data to exercise the application thoroughly as well as
without redundancies.

We have built a system to partially fulfil this vision.  In the rest of this
introductory section, we give an overview of our system, describe some of the
challenges in automating test generation for covering business rules, and
briefly summarize our results.

%Randomly generated test data cannot be expected to suffice for enterprise
%applications with complex rules. Also, systematic test-generation approaches
%based on program analysis (\eg \cite{Emmi:2007,Li:2010,Marcozzi:2012,Pan:2011})
%cannot be expected to tackle enterprise applications, which use a mix of
%multiple language and database technologies in their implementation. Moreover,
%these techniques are directed toward attaining simple forms of code coverage,
%such as statement or branch coverage, rather than coverage of complex business
%rules.

\subsection{An Overview}

Enterprise systems of interest to us are transaction-oriented, which means that
they consist of a set of transactions or operations (\eg create a customer, add
an item to an order, and so on) that operate on databases.  A business rule
applies to a particular operation supported by the system.  Formally, a business
rule describes the relation between the database state before and after the
operation.  Figure~\ref{fig:invoice} shows formalization of the business rule
quoted informally at the beginning of the introduction.  It says that the
operation refers to an invoice record, \subject{inv}, and modifies specific
attributes of \subject{inv}.  There are three scenarios that occur in this rule.
The first scenario applies when the customer to which the invoice refers has
balance type \subject{None}; the \textit{precondition} is shown on the left of
the first arrow.  Note that the invoice has references to the customer to which
this invoice pertains and an order created in the system; in a relational
database, these would be foreign keys in the customer and order tables.  The
second scenario applies when the customer has balance type \subject{Credit} and
the credit limit is sufficient to cover the order total.  The third scenario
applies when the customer has balance type \subject{Credit}, but the order total
exceeds the credit limit.  In each scenario, the effect of the operation is to
compute invoice total and update the customer's residual credit limit; this
effect, or \textit{postcondition} is shown to the right of the arrow in each
case.  Note that business rules refer to the state observable at transaction
boundaries; intermediate program states encountered in the implementation while
a transaction is in process, are not important to business rules.

\begin{figure}
\centering
\includegraphics[trim=55 320 286 54,clip,width=\columnwidth]{figs/invoice}
\vspace*{-14pt}
\caption{Business rule for computing invoice amount.}
\label{fig:invoice}
\end{figure}

Covering a business rule means to exercise each of its constituent scenarios,
referred henceforth as \textit{rule parts}.  To cover these three rule parts in
the business rule of Figure~\ref{fig:invoice}, it is necessary to create
(separately) each of the three preconditions and, in each case, verify that the
postcondition holds after the operation has completed.  This brings us to the
main difficulty in creating tests: appropriate database state, such as a
customer with a certain balance type and an order with a certain amount, needs
to be established before any of these scenarios can be exercised.  We showed in
Figure~\ref{fig:jbilling-flow} the steps that would be required to create these
preconditions.  How can we identify these steps, and the data to be entered in
each of these steps automatically to cover a rule part of a business rule?

Our observation is that the steps that are required to drive the database state
to a desired precondition are carried out by operations, and those operations
too would have rules that specify their functionality.  We could then use
business rules as ``state transformers'' and piece together a sequence of
operations to arrive at a desired state.  The advantage of looking at business
rules as state transformers is that we can adapt the technology developed for
test generation on \textit{programs} for the problem at hand.\footnote{\small We
  clarify, though, that business rules are themselves not executable programs;
  rather they only are an abstract description of the functionality of a
  program.}  The disadvantage of relying on business rules to act as state
transformers is that they need to be specified to a certain level of detail for
them to work out as state transformers; this is generally not a big
problem---practioners tend to write business rules with an intention to be
complete---but their intended use as input to test generation process does
increase expectations from the rules and, therefore, from the analysts who write
them.

\subsection{Our Approach and Results}

At a high level, the idea is to use backward analysis to piece together a
sequence of operations to arrive at a desired state.  We look for an operation
whose business rule has a rule part whose postcondition would imply the desired
precondition.  Such an operation, if executed in a way that that specific rule
part applies, would establish the desired state.  The operation may require some
user-provided values, but may partially rely on prior database state. The
process is repeated until no prior database state is assumed---that is, all the
database state is established by operations identified in the process.

Consider the second rule part of the rule shown in Figure~\ref{fig:invoice} for
the operation to generate an invoice. To satisfy the precondition of the rule
part, a customer with balance type \subject{Credit}, and an associated credit
limit, needs to be created first. Then, an order whose total does not exceed the
customer's credit limit needs to be generated, which involves adding items with
suitable prices to the order. Only after this state has been set up, the
operation for invoice generation can be invoked. An operation sequence and test
data (we explain the notation in Section~\ref{sec:approach}) that achieves this
is:

\vspace*{-4pt}%
{\scriptsize
\begin{alltt}
 State st, BalanceType bt = Credit,  
 int crLimit = 100, int price = 20;
 Customer cust = CreateCustomer(st, bt);
 Customer cust1 = AddCreditLimit(cust, int crLimit);
 Order ord = CreateOrder(cust);
 Item item = CreateItem(int price);
 Order ord1 = AddItemToOrder(ord, item);
 Invoice inv = GenerateInvoice(ord1);  
\end{alltt}}%
\vspace*{-5pt}

The idea of backward traversal is definitely not novel; it is reminiscent of
weakest preconditions.  Our contribution is to make this idea work in the
context of business rules.  In general, the space of possible operation
sequences can be large, in which only a few sequences cover the target rule
part. Thus, the challenge is to search this space soundly, but efficiently in a
goal-driven manner. Specifically, our technique builds the sequence
incrementally using constraint solving. If the logical formula for a sequence is
not satisfiable, it extracts the unsatisfied core of the formula and constructs
new candidate sequences by considering only those operations and rule parts
whose postconditions are compatible with the unsatisfied core. In this way---and
using additional optimizations---the technique can prune out large parts of the
search space and efficiently narrow down to the covering sequences.

We also present a notation for capturing business rules formally, which enables
mechanized analysis for test generation. Moreover, prior to test generation,
formally specified rules can be checked for various consistency and completeness
properties.

We have implemented a prototype system, which includes a business rule editor
(an Eclipse plug-in) and automated analyses for rule checking and test
generation. Our preliminary results illustrate the promise of the approach: for
77~rule parts, modeled from three applications, our technique generated covering
sequences and test data for 99\% of the rule parts and missed only one rule part
(which could not be covered because of limitations of the underlying constraint
solver). By comparison, a technique that performs exhaustive (unguided) search
could cover 74\% of the rule parts, although it explored substantially more
candidate sequences than our technique.

In summary the contributions of this paper are as follows:
\begin{itemize}[noitemsep]
\item A notation for describing business rules formally and a set of
  well-formedness properties. We have also developed an editor for the
  language.
\item An algorithm which can mechanically construct test sequences
  that exercise the business rules of a model. The algorithm uses a
  novel optimization to prune the search space. We have implemented
  the algorithm in a tool called \tool{}.
\item To evaluate our approach we have formalized the business rules
  of three enterprise systems and used \tool{} to generate test
  sequences. Using our approach we are able to generate tests for 99\%
  of the rule parts.
\end{itemize}

In Section~\ref{sec:model} we describe our notation for modeling business
rules and present 4 well-formedness properties of
models. In Section~\ref{sec:example} we present the running example
model, JBilling. In Section~\ref{sec:approach} we describe our approach including several
optimzations. In Section~\ref{sec:eval} is the experimental evaluation using
our implementation \tool{} and in Section~\ref{sec:related} we discuss related
research.   
