\section{Introduction}

A business rule articulates some aspect of the expected functional behavior (or a \textit{requirement})
of an enterprise application. Here is a simple business rule that determines how an
invoice total is determined in a billing application:

\begin{quote}
	The \textit{Balance Type} of a customer affects how invoice total is computed; it can be 
	one of the following:

	\textit{None}: The customer's account will not hold a balance; instead all charges accrued 
	in an order will be included in the next invoice;
	
	\textit{Credit}: The customer's account may accrue charges up to the set credit limit. 
	Charges will automatically be paid from the users credit pool until the set limit is reached. 
	Users are responsible for paying their credit debt as well as any overages.
\end{quote}	

We will examine this rule much more closely later; for now, suffice it to say that the requirements 
of an enterprise system are typically captured by a large number (often in hundreds) of business rules
such as this one.

It is reasonable to expect that functional testing of an enterprise system would \textit{cover} its 
business rules, which is to say, testing would exercise each distinct scenario described in each 
rule.\footnote{This is similar to the notion of path coverage in programs, except this is described
at the level of a business rule.}

