
\begin{table*}[t]
\caption{Formal specification of sample business rules in \subject{jBilling}.}
\centering
\tabcolsep=3pt
{\scriptsize
\begin{tabular}{|l|l|l|l|l|l|}
\hline
& & & & \multicolumn{2}{|c|}{Rules $(R)$} \\
\cline{5-6}
\multicolumn{1}{|c|}{Operation} &
\multicolumn{1}{|c|}{Inputs $(I)$} &
\multicolumn{1}{|c|}{Creates $(C)$} &
\multicolumn{1}{|c|}{Modifies $(M)$} &
\multicolumn{1}{|c|}{Description} &
\multicolumn{1}{|c|}{Formal Representation} \\
\hline \hline
Create & \{{\tt State,} & \{{\tt Customer}\} &
\multicolumn{1}{|c|}{$\emptyset$} &
$R_1$: The \textit{credit limit (crLimit)} of newly  &
$r_{1.1}$: $({\tt true}) \Longrightarrow ({\tt cust.state} = {\tt state} \; \wedge$ \\
Customer & {\tt BalanceType}\} & & & created customers should be \textit{zero} &
\hspace*{10pt}$ {\tt cust.balType} = {\tt  balType} \; \wedge {\tt cust.crLimit} = 0)$ \\
\hline
Create & \{{\tt Int}\} & \{{\tt Item}\} & \multicolumn{1}{|c|}{$\emptyset$} &
$R_2$: The price of an item must be greater &
$r_{2.1}$: $({\tt price} > 0) \Longrightarrow ({\tt item.price} = {\tt price})$ \\
Item & & & & than zero & \\
\hline
Create & \{{\tt Customer}\} & \{{\tt Order}\} &
\multicolumn{1}{|c|}{$\emptyset$} &
$R_3$: The \textit{total} of a newly created order &
$r_{3.1}$: $({\tt true}) \Longrightarrow$ $({\tt ord.total} = 0 \wedge {\tt ord.cust} = {\tt cust})$ \\
Order & & & & should be zero & \\
%% \cline{5-6}
%% & & & & Orders created in the month of &
%% $({\tt month} = {\tt Nov}) \Longrightarrow ({\tt ord.extraDiscount} = 1)$ \\
%% & & & & are eligible for a Thanksgiving &
%% $({\tt month} \neq {\tt Nov}) \Longrightarrow ({\tt ord.extraDiscount}
%% = 0)$  \\
%% & & & & discount of 5\% & \\
\hline
Generate & \{{\tt Order}\} & \{{\tt Invoice}\} &
\multicolumn{1}{|c|}{$\emptyset$} &
$R_4$: A customer's balance type determines &
$r_{4.1}$: $({\tt ord.total} > 0 \wedge {\tt ord.cust.balType} = {\tt None}) \Longrightarrow$ \\
Invoice & & & & how the invoice total is computed &
\hspace*{10pt}$({\tt inv.total} = {\tt ord.total})$ \\
& & & & (see complete rule in the Introduction) &
$r_{4.2}$: $({\tt ord.total} > 0 \wedge {\tt ord.cust.balType} = {\tt Credit} \; \wedge$ \\
& & & & &
\hspace*{10pt}${\tt ord.cust.crLimit} \geq {\tt ord.total}) \Longrightarrow$ \\
& & & & &
\hspace*{10pt}$({\tt inv.total} = 0 \; \wedge$ \\
& & & & &
\hspace*{10pt}${\tt ord.cust.crLimit} = {\tt ord.cust.crLimit@} - {\tt ord.total})$ \\
& & & & &
$r_{4.3}$: $({\tt ord.total} > 0 \wedge {\tt ord.cust.balType} = {\tt Credit} \; \wedge$ \\
& & & & &
\hspace*{10pt}${\tt ord.cust.creditLimit} < {\tt ord.total}) \Longrightarrow$ \\
& & & & &
\hspace*{10pt}$({\tt inv.total} = {\tt ord.total} - {\tt ord.cust.crLimit} \; \wedge$ \\
& & & & &
\hspace*{10pt}${\tt ord.cust.crLimit} = 0)$ \\
\cline{5-6}
& & & & $R_5$: If the customer's residence is in &
$r_{5.1}$: $({\tt ord.total} > 0 \wedge {\tt ord.cust.state} = {\tt NY})
\Longrightarrow$ \\ 
& & & & NY \textit{state}, an additional 2\% discount &
\hspace*{10pt}$({\tt inv.total} = {\tt ord.total} * (98 / 100))$ \\ 
& & & & is given while generating invoices &
$r_{5.2}$: $({\tt ord.total} > 0 \wedge {\tt ord.cust.state} = {\tt Other})
\Longrightarrow$ \\
& & & & &
\hspace*{10pt}$({\tt inv.total} = {\tt ord.total})$ \\
\hline
Add Credit & \{{\tt Customer,} & \multicolumn{1}{|c|}{$\emptyset$} &
\{{\tt Customer}\} &
$R_6$: The credit limit can be incremented for &
$r_{6.1}$: $({\tt cust.balType} = {\tt Credit} \wedge {\tt amount} > 0) \Longrightarrow$ \\
Limit & {\tt Int}\} & & & customers with balance type \textit{Credit} &
\hspace*{10pt}$({\tt cust.crLimit} = {\tt cust.crLimit@} + {\tt amount})$\\
\hline
%% Deactivate & \{{\tt Customer}\} & \multicolumn{1}{|c|}{$\emptyset$} &
%% \{{\tt Customer}\} &
%% Active customers can be deactivated &
%% $({\tt cust.status} = {\tt Active}) \Longrightarrow ({\tt cust.status} = {\tt Inactive})$ \\
%% Customer & & & & & \\
%% \hline
Add Item & \{{\tt Order}, {\tt Item}\} &
\multicolumn{1}{|c|}{$\emptyset$} & \{{\tt Order}\} &
$R_7$: Adding an item to an order increases &
$r_{7.1}$: $({\tt true}) \Longrightarrow ({\tt ord.total} = {\tt ord.total@} +
{\tt item.price})$ \\
to Order & & & & the order's total by the item's price & \\
\hline
\end{tabular}
}
\vspace*{-3ex}
\label{tab:bookstore-rules-spec}
\end{table*}

\section{Example Application Model}
\label{sec:example}

Before presenting our technique, we elaborate upon the billing application
example (\subject{jBilling}) mentioned in the previous sections: we explain
various operations and rules for the application, which we use subsequently to
illustrate our technique. To facilitate the illustration, we use a simplified
version inspired by the actual application~\cite{jbilling}. (In the empirical
evaluation, we modeled a larger part of the application.) The model for
\subject{jBilling} includes four entities: \subject{Customer}, \subject{Item},
\subject{Order}, and \subject{Invoice}.  Figure~\ref{fig:sample-app} shows the
flow among some of the operations, which create, read, or modify these entities.
%For example, operation
%\subject{CreateCustomer} creates an instance of \subject{Customer}, whereas the
%operation \subject{AddItemToOrder} modifies an \subject{Order} instance by
%adding a new item to the order.

Table~\ref{tab:bookstore-rules-spec} presents the formal specification of the
\subject{jBilling} operations and rules. Columns~2--4 list, respectively, the
entities read, created, and modified by an operation. Columns~5 and 6 provide
informal descriptions and formal representations of the rules associated with an
operation.  \subject{GenerateInvoice} has two rules associated with it, whereas
all the other operations have only one rule associated with them. The first rule
for \subject{GenerateInvoice} has three rule parts, which determine how the
invoice total and the customer's credit limit are updated by the operation,
based on the customer's balance type.  The second rule for
\subject{GenerateInvoice} has two rule parts---which also determine the
computation of invoice total, in this case, based on the customer's state of
residence.

\begin{figure}[t]
\centering
\includegraphics[trim=47 420 380 42,clip,width=.8\columnwidth]{figs/appModel.pdf}
\vspace*{-12pt}
\caption{Sample operations and their interactions in \textit{jBilling}.}
\vspace*{-0pt}
\label{fig:sample-app}
\end{figure}
%\vspace*{-2pt}

The boolean expressions in the rule parts refer to the input/crea\-ted/modified
entities by names: \eg the rule part for \subject{CreateOrder} refers to the
input \subject{Customer} instance as \subject{cust} and the created
\subject{Order} instance as \subject{ord}. The expressions also refer to
attributes of the entities. For instance, some of the relevant attributes of
\subject{Customer} are: \subject{State: state}, \subject{BalanceType: balType},
and \subject{int: crLimit}, 
%\vspace*{-3pt}
%{\scriptsize
%\begin{verbatim}
  %Customer {
      %State: state
      %BalanceType: balType
      %int: crLimit
  %}
%\end{verbatim}
%}
%\vspace*{-7pt}
where \subject{State} and \subject{BalanceType} are
enumerated types (\eg \subject{BalanceType} can take two values---\subject{None}
or \subject{Credit}) and \subject{crLimit} is an integer.

If an expression needs to refer to the old and new values of an attribute (the
values prior and subsequent to an operation invocation), the old value is
distinguished by appending `@' to the attribute name. For instance, consider the
postcondition in the rule for \subject{AddCreditLimit}: {\small ${\tt
    cust.crLimit} = {\tt cust.crLimit@} + {\tt amount}$}; this states that the
input credit amount is added to the customer's old credit limit to obtain the
new credit limit.
