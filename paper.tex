\documentclass{sig-alternate}


\usepackage{amssymb}
\usepackage{verbatim}
\usepackage{times}
\usepackage{graphicx}
\usepackage{epsfig}
\usepackage{url}
\usepackage{txfonts}
\usepackage{xspace}
\usepackage{float}
\usepackage{latexsym}
%\usepackage{natbib}
\usepackage{alltt}
\usepackage{color}
\usepackage{textcomp}
\usepackage{balance}
%\usepackage[pass,letterpaper]{geometry}
\usepackage[ruled,vlined]{algorithm2e}

\setlength{\textfloatsep}{0.4\baselineskip}

\begin{document}

\newcommand{\ie}{\textit{i.e.,} }
\newcommand{\eg}{\textit{e.g.,} }
\newcommand{\subject}[1]{\texttt{\small #1}}
\newcommand{\rules}{{\mathcal R}}
\newcommand{\tests}{{\mathcal T}}
\newcommand{\elemexists}{\mt{exists}}
\newcommand{\elemenabled}{\mt{enabled}}
\newcommand{\elemexplored}{\mt{explored}}
\newcommand{\note}[1]{{\color{red}$[$ \bf #1 $]$}}
\newcommand{\lang}[1]{\texttt{\scriptsize #1}}
\newcommand{\tool}{\textsc{buster}}
\newcommand{\exhaust}{\textsc{exhaust}}
\newcommand{\choco}{\textsc{choco}}

\title{Test Data Generation from Business Rules}

\author{Simon H Jensen \and Suresh Thummalapenta \and Saurabh Sinha \and Satish Chandra}

\maketitle

\begin{abstract}
Enterprise applications are difficult to test because their intended
functionality is either not described precisely enough or described in
cumbersome business rules. It takes a lot of effort on the part of a test
architect to understand all the business rules and design tests that ``cover''
them, \ie exercise all their constituent scenarios. Part of the problem is that
it takes a complicated set up sequence to drive an application to a state in
which a business rule can even fire.  In this paper, we present a business rule
modeling language that can be used to capture functional specification of an
enterprise system. The language makes it possible to build tool support for rule
authoring, so that obvious deficiencies in rules can be detected
mechanically. Most importantly, we show how to mechanically generate test
sequences---\ie test steps and test data---needed to exercise these business
rules. To this end, we translate the rules into logical formulae and use
constraint solving to generate test sequences.  One of our contributions is to
overcome scalability issues in this process, and we do this by using a novel
algorithm for organizing search through the space of candidate sequences to
discover covering sequences.  Our results on three case studies show the promise
of our approach.
\end{abstract}

\section{Introduction}

A business rule articulates some aspect of the expected functional behavior (or
a \textit{requirement}) of an enterprise application. Here is a simple business
rule that determines how an invoice total is determined in a billing
application:
%
\begin{quote}
	The \textit{Balance Type} of a customer affects how invoice total is computed; it can be 
	one of the following:

	\textit{None}: The customer's account will not hold a balance; instead all charges accrued 
	in an order will be included in the next invoice;
	
	\textit{Credit}: The customer's account may accrue charges up to the set credit limit. 
	Charges will automatically be paid from the users credit pool until the set limit is reached. 
	Users are responsible for paying their credit debt as well as any overages.
\end{quote}	
%
We will examine this rule closely later; for now, suffice it to say that the
requirements of an enterprise system are typically captured by a large number
(often, hundreds) of business rules such as the one above.

It is reasonable to expect functional testing of an enterprise system to
\textit{cover} its business rules, which is to say, testing would exercise every
distinct scenario described in each of its business rules.  For example, in the
preceding rule, one of the scenario to be exercised is that a customer's balance
type is credit, and that the order amount exceeds the customer's credit limit.
A test that exercises this scenario would set up a customer with the balance
type as \textit{credit} as well as a certain credit limit, say, \textit{100},
create an order and add items to the order to bring a total to, say,
\textit{120}, to exceed the credit limit for that customer, and then finally
create an invoice for that order,and verify the invoiced amount.
Figure~\ref{fig:jbilling-flow} illustrates this flow.  Although the values
\textit{credit}, \textit{100}, and \textit{120} can be identified just from this
rule (by constraint solving), identifying a test sequence is also important to
apply those values at the right fields on the appropriate
screens.\footnote{\small Not all fields on the screens in
  Figure~\ref{fig:jbilling-flow} are constrained from the point of view of
  exercising a particular scenario, but the application might still demand
  sensible values for them. The tester is expected to make these up.}

\begin{figure*}
\centering
\includegraphics[trim=47 210 35 50,clip,width=\textwidth]{figs/jbilling-flow}
\vspace*{-10pt}
\caption{Test sequence for exercising a business rule from the
  \subject{jBilling} application.}
\vspace*{-7pt}
\label{fig:jbilling-flow}
\end{figure*}

It requires a carefully thought-out test scenario, \ie a sequence of test steps
as well as associated test data, \ie values to be entered in the relevant test
steps to exercise a business rule, or a scenario thereof.  Without systematic
test creation, testers may end up creating multiple tests that exercise the same
business rule, or a scenario thereof, over and over again without any additional
benefit (especially if they are incentivised by the number of tests created
rather than quality of the created tests), and more problematically, may neglect
to create tests for some other business rules.

In practice, due to time pressures, testers are more often ad-hoc than
methodical in creating test scenarios and test data.  One part of the problem is
that a realistic system might have hundreds of business rules written out in
plain text, and it is difficult to grasp a global view of how the rules together
describe the application behavior.  A related part of the problem is that it may
need complex reasoning to piece together test sequences that would cover each
scenario.  The net result is that, despite a lot of resources spent in testing,
bugs still escape into the field.

\hyphenation{non-prog-ram-mers}

Our vision is to make testing of enterprise software more tool-based, by
adapting technology developed for automated and systematic test generation for
programs.  In this vision, business rules would be written in a structured
notation that allows mechanized analysis.  Special editors could be created
to enable non-programm\-ers to capture business rules in a structured notation;
this is an independent challenge in \textit{end-user} programming.  A tool would
validate business rules and point out any ambiguities or omissions that it can
detect.  After the business rules pass validation, another tool would generate
test sequences and test data to exercise the application thoroughly as well as
without redundancies.

We have built a system to partially fulfil this vision.  In the rest of this
introductory section, we give an overview of our system, describe some of the
challenges in automating test generation for covering business rules, and
briefly summarize our results.

%Randomly generated test data cannot be expected to suffice for enterprise
%applications with complex rules. Also, systematic test-generation approaches
%based on program analysis (\eg \cite{Emmi:2007,Li:2010,Marcozzi:2012,Pan:2011})
%cannot be expected to tackle enterprise applications, which use a mix of
%multiple language and database technologies in their implementation. Moreover,
%these techniques are directed toward attaining simple forms of code coverage,
%such as statement or branch coverage, rather than coverage of complex business
%rules.

\subsection{An Overview}

Enterprise systems of interest to us are transaction-oriented, which means that
they consist of a set of transactions or operations (\eg create a customer, add
an item to an order, and so on) that operate on databases.  A business rule
applies to a particular operation supported by the system.  Formally, a business
rule describes the relation between the database state before and after the
operation.  Figure~\ref{fig:invoice} shows formalization of the business rule
quoted informally at the beginning of the introduction.  It says that the
operation refers to an invoice record, \subject{inv}, and modifies specific
attributes of \subject{inv}.  There are three scenarios that occur in this rule.
The first scenario applies when the customer to which the invoice refers has
balance type \subject{None}; the \textit{precondition} is shown on the left of
the first arrow.  Note that the invoice has references to the customer to which
this invoice pertains and an order created in the system; in a relational
database, these would be foreign keys in the customer and order tables.  The
second scenario applies when the customer has balance type \subject{Credit} and
the credit limit is sufficient to cover the order total.  The third scenario
applies when the customer has balance type \subject{Credit}, but the order total
exceeds the credit limit.  In each scenario, the effect of the operation is to
compute invoice total and update the customer's residual credit limit; this
effect, or \textit{postcondition} is shown to the right of the arrow in each
case.  Note that business rules refer to the state observable at transaction
boundaries; intermediate program states encountered in the implementation while
a transaction is in process, are not important to business rules.

\begin{figure}
\centering
\includegraphics[trim=55 320 286 54,clip,width=\columnwidth]{figs/invoice}
\vspace*{-14pt}
\caption{Business rule for computing invoice amount.}
\label{fig:invoice}
\end{figure}

Covering a business rule means to exercise each of its constituent scenarios,
referred henceforth as \textit{rule parts}.  To cover these three rule parts in
the business rule of Figure~\ref{fig:invoice}, it is necessary to create
(separately) each of the three preconditions and, in each case, verify that the
postcondition holds after the operation has completed.  This brings us to the
main difficulty in creating tests: appropriate database state, such as a
customer with a certain balance type and an order with a certain amount, needs
to be established before any of these scenarios can be exercised.  We showed in
Figure~\ref{fig:jbilling-flow} the steps that would be required to create these
preconditions.  How can we identify these steps, and the data to be entered in
each of these steps automatically to cover a rule part of a business rule?

Our observation is that the steps that are required to drive the database state
to a desired precondition are carried out by operations, and those operations
too would have rules that specify their functionality.  We could then use
business rules as ``state transformers'' and piece together a sequence of
operations to arrive at a desired state.  The advantage of looking at business
rules as state transformers is that we can adapt the technology developed for
test generation on \textit{programs} for the problem at hand.\footnote{\small We
  clarify, though, that business rules are themselves not executable programs;
  rather they only are an abstract description of the functionality of a
  program.}  The disadvantage of relying on business rules to act as state
transformers is that they need to be specified to a certain level of detail for
them to work out as state transformers; this is generally not a big
problem---practioners tend to write business rules with an intention to be
complete---but their intended use as input to test generation process does
increase expectations from the rules and, therefore, from the analysts who write
them.

\subsection{Our Approach and Results}

At a high level, the idea is to use backward analysis to piece together a
sequence of operations to arrive at a desired state.  We look for an operation
whose business rule has a rule part whose postcondition would imply the desired
precondition.  Such an operation, if executed in a way that that specific rule
part applies, would establish the desired state.  The operation may require some
user-provided values, but may partially rely on prior database state. The
process is repeated until no prior database state is assumed---that is, all the
database state is established by operations identified in the process.

Consider the second rule part of the rule shown in Figure~\ref{fig:invoice} for
the operation to generate an invoice. To satisfy the precondition of the rule
part, a customer with balance type \subject{Credit}, and an associated credit
limit, needs to be created first. Then, an order whose total does not exceed the
customer's credit limit needs to be generated, which involves adding items with
suitable prices to the order. Only after this state has been set up, the
operation for invoice generation can be invoked. An operation sequence and test
data (we explain the notation in Section~\ref{sec:approach}) that achieves this
is:

\vspace*{-4pt}%
{\scriptsize
\begin{alltt}
 State st, BalanceType bt = Credit,  
 int crLimit = 100, int price = 20;
 Customer cust = CreateCustomer(st, bt);
 Customer cust1 = AddCreditLimit(cust, int crLimit);
 Order ord = CreateOrder(cust);
 Item item = CreateItem(int price);
 Order ord1 = AddItemToOrder(ord, item);
 Invoice inv = GenerateInvoice(ord1);  
\end{alltt}}%
\vspace*{-5pt}

The idea of backward traversal is definitely not novel; it is reminiscent of
weakest preconditions.  Our contribution is to make this idea work in the
context of business rules.  In general, the space of possible operation
sequences can be large, in which only a few sequences cover the target rule
part. Thus, the challenge is to search this space soundly, but efficiently in a
goal-driven manner. Specifically, our technique builds the sequence
incrementally using constraint solving. If the logical formula for a sequence is
not satisfiable, it extracts the unsatisfied core of the formula and constructs
new candidate sequences by considering only those operations and rule parts
whose postconditions are compatible with the unsatisfied core. In this way---and
using additional optimizations---the technique can prune out large parts of the
search space and efficiently narrow down to the covering sequences.

We also present a notation for capturing business rules formally, which enables
mechanized analysis for test generation. Moreover, prior to test generation,
formally specified rules can be checked for various consistency and completeness
properties.

We have implemented a prototype system, which includes a business rule editor
(an Eclipse plug-in) and automated analyses for rule checking and test
generation. Our preliminary results illustrate the promise of the approach: for
77~rule parts, modeled from three applications, our technique generated covering
sequences and test data for 99\% of the rule parts and missed only one rule part
(which could not be covered because of limitations of the underlying constraint
solver). By comparison, a technique that performs exhaustive (unguided) search
could cover 74\% of the rule parts, although it explored substantially more
candidate sequences than our technique.

In summary the contributions of this paper are as follows:
\begin{itemize}[noitemsep]
\item A notation for describing business rules formally and a set of
  well-formedness properties. We have also developed an editor for the
  language.
\item An algorithm which can mechanically construct test sequences
  that exercise the business rules of a model. The algorithm uses a
  novel optimization to prune the search space. We have implemented
  the algorithm in a tool called \tool{}.
\item To evaluate our approach we have formalized the business rules
  of three enterprise systems and used \tool{} to generate test
  sequences. Using our approach we are able to generate tests for 99\%
  of the rule parts.
\end{itemize}

In Section~\ref{sec:model} we describe our notation for modeling business
rules and present 4 well-formedness properties of
models. In Section~\ref{sec:example} we present the running example
model, JBilling. In Section~\ref{sec:approach} we describe our approach including several
optimzations. In Section~\ref{sec:eval} is the experimental evaluation using
our implementation \tool{} and in Section~\ref{sec:related} we discuss related
research.   


\newcommand{\term}{\textit}
\newcommand{\lit}{\texttt}

\section{Rule Modeling and Checking}
\label{sec:model}

In this section, we discuss the notation for modeling business rules and the
static checking for completeness and consistency performed on formally captured
rules.

\begin{figure}[t]
\centering
{\small
\tabcolsep=3pt
\begin{tabular}{lll}
\term{RuleSpec} & ::= & \term{Entities} \term{Operations} \\
\\
\term{Entities} & ::= & \term{Entity} \term{Entities} | $\epsilon$ \\
\term{Entity} & ::= & \term{Enum} | \term{Object} \\
\term{Enum} & ::= & \lit{enum} \term{ID} \{ \term{EnumVals} \} \\
\term{EnumVals} & ::= & \term{ID} \term{EnumVals} | $\epsilon$ \\
\term{Object} & ::= & \lit{object} \term{ID} \{ \term{VarDecl} \} \\
\\
\term{Operations} & ::= & \term{Operation} \term{Operations} | $\epsilon$ \\
\term{Operation} & ::= & \lit{operation} \term{ID} \{ \term{Input}
\term{Creates} \term{Modifies} \term{Rules} \term{Triggers} \} \\
\term{Input} & ::= & \lit{input} : \term{VarDecl} \\
\term{Creates} & ::= & \lit{creates} : \term{VarDecl} \\
\term{Modifies} & ::= & \lit{modifies} : \term{VarDecl} \\
\term{Rules} & ::= & \term{Rule} \term{Rules} | $\epsilon$ \\
\term{Rule} & ::= & \lit{group} \term{ID} \{ \term{RuleParts} \} \\
\term{RuleParts} & ::= & \term{RulePart} \term{RuleParts} | $\epsilon$ \\
\term{RulePart} & ::= & \lit{rule} \term{ID} \{ \lit{pre} : \term{Expr} ;
\lit{post} : \term{Expr} \} \\
\term{Triggers} & ::= & \lit{next} : \term{ID} | $\epsilon$ \\
\\
\term{VarDecl} & ::= & \term{TypeName} : \term{ID} \term{VarDecl} | $\epsilon$
\\
\term{TypeName} & ::= & \lit{int} | \lit{bool} | \lit{float} | \lit{string} |
\lit{set<\term{\textrm{TypeName}}>} | \term{ID} \\
\term{Expr} & ::= & \\
\end{tabular}
}
\caption{Partial rule-modeling syntax.}
\label{fig:model-syntax}
\end{figure}

\subsection{Rule-Modeling Language}

Overall, our approach models rules in the context of operations in the system
under test.  An operation has a set of input entities, a set of created
entities, a set of modified entities, and a set of rules, where each rule
consists of a set of precondition-postcondition pairs. For example,
Figure~\ref{fig:invoice} shows the operation for computing invoice totals in the
\subject{JBilling} application, which takes as input an invoice and modifies
certain attributes. The rules associated with this operation, which govern how
the invoice total and the customer's credit limit are updated, are modeled with
the operation in the form of precondition-postcondition pairs.

Figure~\ref{fig:model-syntax} presents the formal syntax of the rule-modeling
language (for clarity, we omit some of the details and present only the
important parts of the language). A \textit{rule specification} consists of
entities and operations. An \textit{entity} can be an object in the system (\eg
invoice, order, customer) or an enumerated type (\eg a customer's balance type
can be \subject{None}, \subject{Credit}, or \subject{Prepaid}).

The key part of the syntax, which models rules, is based on
operations. Formally, an \textit{operation} ${\cal O}$ is the tuple $(I, C, M,
R, o_t)$, where $I$ is the set of input entities read during the execution of
${\cal O}$, $C$ is the set of entities created by ${\cal O}$, $M$ is the set of
entities whose attributes are modified by ${\cal O}$, $R$ is the set of rules
that describe the behavior of ${\cal O}$, and $o_t$ is an operation, possibly
none, that is automatically triggered in the system after the execution of
${\cal O}$.

A \textit{rule} $R = \{r_1, r_2, \ldots, r_k\}, k \geq 1,$ consists of a set of
rule parts. A \textit{rule part} $r$ is a precondition-postcondition pair, $p
\Longrightarrow q$, where $p$ and $q$ are boolean formulas such that if $p$
holds in the state before the operation, $q$ is true in the state resulting from
the execution of the operation. If the precondition of a rule part is true, we
say that the rule is \textit{applicable}.

Consider the rule illustrated in Figure~\ref{fig:invoice} for the
\subject{Compute Invoice Total} operation. The rule has three rule parts, each
of which consists of a precondition and a postcondition. The first rule part
pertains to the case where the customer's balance type is \subject{None}; the
second rule part is for the case where the balance type is \subject{Credit} and
the credit limit exceeds or equals the order total; the third rule part covers
the case where the balance type is \subject{Credit} and the order total exceeds
the credit limit.

Often in enterprise systems, the execution of an operation or a transaction
automatically triggers other transactions or operations. Our rule-modeling
syntax accommodates this. For example, ...

\subsection{Rule Checking}
 
We impose a few well-formedness constraints on rule specifications to ensure
consistency and completeness, and accuracy of operation chaining for test
generation. These constraints are amenable to automated static checking. Thus,
we envision that the rules can be iteratively refined in a rule editor, based on
automatic (semantic) checking for constraint violations (in addition to
syntactic checking for conformance to the modeling syntax).

\paragraph*{Property 1: Rule-part Disjointedness}
In our notation, a rule part is intended to represent disjoint preconditions so
that when a rule is applicable, the precondition of only one rule part is true;
consequently, there is no ambiguity in identifying the relevant rule part for an
applicable rule. Formally, we define this property as follows. Let $R= \{r_1,
r_2, \ldots, r_k\}$ be a rule such that $k \geq 2$. Then, for all $r_i, r_j \in
R$ where $ r_i \coloneqq (p_i \Longrightarrow q_i)$ and $r_j \coloneqq (p_j
\Longrightarrow q_j)$, $(p_i \wedge p_j)$ must not be satisfiable. A simple
example of a rule specification that violates this property is $p_i = (a > 0)$
and $p_j = (a < 10)$; this specification represents ambiguous behavior when, for
example, $a = 5$.  The rule parts illustrated in Figure~\ref{fig:invoice} have
disjoint preconditions.

To check that a rule satisfies this property, first, we enumerate all pairs of
rule parts for the rule. Then, for each pair of rule parts (with preconditions
$p_i$ and $p_j$), we determine whether the boolean formula $(p_i \wedge p_j)$
has a solution; if it does, we flag a violation of the property.

\paragraph*{Property 2: Rule-part Completeness}
This property is intended to ensure that a rule specifies the complete operation
behavior for the variables mentioned in the rule. Let $R= \{r_1, r_2, \ldots,
r_k\}$ be a rule such that $k \geq 2$ and $r_i \coloneqq (p_i \Longrightarrow
q_i)$. Then, $\neg(p_1 \vee p_2 \vee \ldots \vee p_k)$ must not be
satisfiable. For example, the rule consisting of two rule parts with
preconditions $(a < 5)$ and $(a > 10)$, respectively, violates the completeness
property because the operation behavior for $5 \leq a \leq 10$ is left
unspecified.

To verify this property, our technique checks, for each rule, whether the
formula $\neg(p_1 \vee p_2 \vee \ldots \vee p_k)$ has a solution.

%% \paragraph*{Postcondition uniformness property}
%% This property requires that all postconditions for a rule must assert values for
%% the same set of variables, which ensures that the effect of an operation is
%% consistently specified irrespective of which part of a rule causes the rule to
%% be applicable. \textbf{(Do we need this? Fig 2 violates it.)}

\paragraph*{Property 3: Rule Compatibility}
The rule compatibility property requires that for any set of applicable rules of
an operation, the postconditions of their relevant rule parts must not be
conflicting. To illustrate, consider rules $R_1 = \{r_1\}$ and $R_2 = \{r_2\}$
for an operation, such that $r_1 \coloneqq ((a > 0) \Longrightarrow (total =
10))$, $r_2 \coloneqq ((b > 0) \Longrightarrow (total = 20))$, and the two
preconditions are not disjoint (\ie $(a > 0) \wedge (b > 0)$ is
satisfiable). This pair of rules violates the compatibility property because the
postconditions are conflicting, whereas the corresponding preconditions can be
true simultaneously. In general, let $r_1 \coloneqq (p_1 \Longrightarrow q_1)$
and $r_2 \coloneqq (p_2 \Longrightarrow q_2)$ be the relevant rule parts of two
applicable rules of an operation. Then, if $(p_1 \wedge p_2)$ is satisfiable,
$(q_1 \wedge q_2)$ must be satisfiable.

If two rules violate this property, it might in fact indicate that their rule
parts can be merged into one rule. In the preceding example, $R_1$ and $R_2$
could be merged into one rule with two rule parts $R_m = \{r_{m_1}, r_{m_2}\}$,
where $r_{m_1} \coloneqq (a > 0) \Longrightarrow (total = 10)$ and $r_{m_2}
\coloneqq (a \leq 0 \wedge b > 0) \Longrightarrow (total = 20)$. Note adding the
conjunct $a \leq 0$ (the negation of the precondition of $r_1$) to the
precondition of $r_{m_2}$ makes the two preconditions disjoint, which ensures
that $R_m$ satisfies the rule-part disjointedness property. Alternatively, the
negation of the precondition of $r_2$ could be added to the precondition of
$r_{m_1}$ to satisfy this property.

To verify this property, ...

\paragraph*{Property 4: Rule Dependence}
Finally, we enforce the restriction that there can be no data dependence between
the postcondition of one rule and the precondition of another rule of the same
operation. This property ensures that there is no implicit ordering among the
rules of an operation. When such an ordering exists between two rules of an
operation, the operation should be split into two operations. We formalize this
property as follows. Let ${\cal R} = \{R_1, R_2, \ldots, R_n\}$ be the rules
associated with an operation. For any $R_i, R_j \in {\cal R}$, let $V_{i,
  \mathit{post}}$ be the set of variables used in the postconditions of the rule
parts of $R_i$ and $V_{j, \mathit{pre}}$ be the set of variables used in the
preconditions of the rule parts of $R_j$. Then, $V_{i, \mathit{post}} \cap V_{j,
  \mathit{pre}} = \emptyset$.

\section{Example Model}

\begin{figure*}
\centering
\includegraphics[trim=45 435 40 38,clip,width=7.5in]{figs/appModel.pdf}
\caption{Operations and their interactions with entities of the sample billing application.}
\label{fig:sample-app}
\end{figure*}

We next introduce a sample billing application (along with its operations and
rules) that creates orders for customers and generates invoices. We use this
application as an illustrative example to explain our technique. The model
includes four entities: \textit{Customer}, \textit{Item}, \textit{Order}, and
\textit{Invoice}.  Figure~\ref{fig:sample-app} shows all operations in the
application and their interactions (create or modify) with the entities.  For
example, the operation \textit{CreateCustomer} creates an instance of
\textit{Customer}, whereas the operation \textit{AddItemToOrder} modifies an
\textit{Order} instance by adding a new item to the order.

\begin{table*}[t]
\caption{Formal specification of the business rules in the sample billing application.}
\centering
\tabcolsep=4pt
{\scriptsize
\begin{tabular}{|l|l|l|l|l|l|}
\hline
& & & & \multicolumn{2}{|c|}{Rules $(R)$} \\
\cline{5-6}
\multicolumn{1}{|c|}{Operation} &
\multicolumn{1}{|c|}{Inputs $(I)$} &
\multicolumn{1}{|c|}{Creates $(C)$} &
\multicolumn{1}{|c|}{Modifies $(M)$} &
\multicolumn{1}{|c|}{Description} &
\multicolumn{1}{|c|}{Formal Representation} \\
\hline \hline
Create & \{{\tt State}\} & \{{\tt Customer}\} &
\multicolumn{1}{|c|}{$\emptyset$} &
The \textit{status} of newly created customers &
$({\tt true}) \Longrightarrow ({\tt cust.state} = {\tt state} \; \wedge$ \\
Customer& & & & should be \textit{Inactive} until they are &
\hspace*{10pt}${\tt cust.status} = {\tt Inactive})$ \\
& & & & verified and activated &  \\
\hline
Create & \{{\tt Customer}\} & \{{\tt Order}\} &
\multicolumn{1}{|c|}{$\emptyset$} &
New orders can be created only for &
$({\tt cust.status} = {\tt Active}) \Longrightarrow$ \\
Order & & & & customers whose \textit{status} is \textit{Active} &
\hspace*{10pt}$({\tt ord.total} = 0 \wedge
{\tt ord.cust} = {\tt cust})$ \\
%% \cline{5-6}
%% & & & & Orders created in the month of &
%% $({\tt month} = {\tt Nov}) \Longrightarrow ({\tt ord.extraDiscount} = 1)$ \\
%% & & & & are eligible for a Thanksgiving &
%% $({\tt month} \neq {\tt Nov}) \Longrightarrow ({\tt ord.extraDiscount}
%% = 0)$  \\
%% & & & & discount of 5\% & \\
\hline
Generate & \{{\tt Order}\} & \{{\tt Invoice}\} &
\multicolumn{1}{|c|}{$\emptyset$} &
A customer's balance type determines &
$({\tt ord.total} > 0 \wedge {\tt ord.cust.balType} = {\tt None}) \Longrightarrow$ \\
Invoice & & & & how the invoice total is computed &
\hspace*{10pt}$({\tt inv.total} = {\tt ord.total})$ \\
& & & & (see complete rule in the Introduction) &
$({\tt ord.total} > 0 \wedge {\tt ord.cust.balType} = {\tt Credit} \; \wedge$ \\
& & & & &
\hspace*{10pt}${\tt ord.cust.crLimit} \geq {\tt ord.total}) \Longrightarrow$ \\
& & & & &
\hspace*{10pt}$({\tt inv.total} = 0 \; \wedge$ \\
& & & & &
\hspace*{10pt}${\tt ord.cust.crLimit'} = {\tt ord.cust.crLimit} - {\tt ord.total})$ \\
& & & & &
$({\tt ord.total} > 0 \wedge {\tt ord.cust.balType} = {\tt Credit} \; \wedge$ \\
& & & & &
\hspace*{10pt}${\tt ord.cust.creditLimit} < {\tt ord.total}) \Longrightarrow$ \\
& & & & &
\hspace*{10pt}$({\tt inv.total} = {\tt ord.total} - {\tt ord.cust.crLimit} \; \wedge$ \\
& & & & &
\hspace*{10pt}${\tt ord.cust.crLimit} = 0)$ \\
\cline{5-6}
& & & & If the customer's residence is in &
$({\tt ord.total} > 0 \wedge {\tt ord.cust.state} = {\tt NY})
\Longrightarrow$ \\ 
& & & & NY \textit{state}, an additional 2\% discount &
\hspace*{10pt}$({\tt inv.total} = {\tt ord.total} * (98 / 100))$ \\ 
& & & & is given while generating invoices &
$({\tt ord.total} > 0 \wedge {\tt ord.cust.state} = {\tt Other})
\Longrightarrow$ \\
& & & & &
\hspace*{10pt}$({\tt inv.total} = {\tt ord.total})$ \\
\hline
Activate & \{{\tt Customer}\} & \multicolumn{1}{|c|}{$\emptyset$} &
\{{\tt Customer}\} &
Inactive customers can be activated &
$({\tt cust.status} = {\tt Inactive}) \Longrightarrow ({\tt cust.status} = {\tt Active})$ \\
Customer & & & & & \\
\hline
Deactivate & \{{\tt Customer}\} & \multicolumn{1}{|c|}{$\emptyset$} &
\{{\tt Customer}\} &
Active customers can be deactivated &
$({\tt cust.status} = {\tt Active}) \Longrightarrow ({\tt cust.status} = {\tt Deactive})$ \\
Customer & & & & & \\
\hline
Add Item & \{{\tt Order}, {\tt Item}\} &
\multicolumn{1}{|c|}{$\emptyset$} & \{{\tt Order}\} &
Adding an item to an order increases &
$({\tt true}) \Longrightarrow ({\tt ord.total'} = {\tt ord.total} +
{\tt item.price})$ \\
to Order & & & & the order's total by the item's price & \\
\hline
\end{tabular}
}
\vspace*{-10pt}
\label{tab:bookstore-rules-spec}
\end{table*}

Table~\ref{tab:bookstore-rules-spec} shows the business rules in each
operation. The table provides both an informal description of each rule and its
formal representation expected by our technique.  \note{add formal
  representation of the rules and describe the rules along with their rule part}

\section{Test data generation}

In this section we present the algorithm that generates test data
from a model written in the modeling language defined in
Section~\ref{sec:model}. 

At its core the algorithm uses backwards substitution to generate a
sequence of operations that puts the application into the desired
state needed to cover a given a target operation. 

\subsection{Control Flow Graph Representation of Models}
\label{sec:control-flow-graph}

Before the actual backwards substitution takes place the model is
converted into a control flow graph(CFG) representation. The process of
converting a single operation to it's CFG is
illustrated in Figure~\ref{fig:cfg}. As the preconditions of all rules in
a group are disjoint, control flow through a single group can cover
exactly one precondition-post condition pair. The CFG reflects this by
representing a single group as a choice between the available rules, as
seen in Figure~\ref{fig:cfgb}. Given a CFG component for each group in
the operation, we model the entire operation by sequencing all
components (Figure~\ref{fig:cfgc}). As per well-formedness property 4
there is data dependence between rules in an operation and therefore
all orderings are equivalent.

\begin{figure*}
%  \centering
  \begin{subfigure}[b]{0.4\textwidth}
    \centering
    \includegraphics[width=.4\linewidth]{figs/cfg-example1}
    \caption{An operation}
    \label{fig:cfga}
  \end{subfigure}%
  \begin{subfigure}[b]{0.4\textwidth}
    \centering
    \includegraphics[width=.4\linewidth]{figs/cfg-example2}
    \caption{The interopertational flow for group 2}
    \label{fig:cfgb}
  \end{subfigure}%
  \begin{subfigure}[b]{0.2\textwidth}
    \centering
    \includegraphics[width=.4\linewidth]{figs/cfg-example3}
    \caption{The CFG component for the entire operation}
    \label{fig:cfgc}
  \end{subfigure}%
  \caption{Control flow graph generation for a single operation}
  \label{fig:cfg}
\end{figure*}

The interoperational flow is defined by the input/output relations
between operations. An operation creating an entity will naturally be
succeeded by any operation modifying the entity. These edges can either
be setup initially or created on demand during backwards substitution. 

Some edges are specified directly by the designer of the model...next
and triggers

\subsection{Backwards substitution}
\label{sec:backw-subst}

When sequencing two operations we need to substitute the entities
consumed by the successor operation by the entities created or
modified by the predecessor operation.

In our algorithm we move this problem to the underlying solver by
creating constraints that bind the identifiers in the post condition
of the predecessor operation to the identifiers of same type on the
pre condition of the successor operation. This binding must account
for objects which have subfields and create bindings for those as
well.



\section{Empirical Evaluation}
\label{sec:eval}

We implemented our technique in a prototype tool called \tool{} (BUSiness TEsting Rules),
and conducted two empirical studies using two open-source applications 
and one IBM proprietary application. In the first study, we compared
the effectiveness of \tool{} in covering business rules with 
a related technique that systematically explores search space 
without any guidance. In the second study, we investigated the efficiency
of both the techniques. After describing the experimental setup,
we present results of the two studies.

\begin{table}[t]
\caption{Subjects used in our empirical studies.}
\centering
{\scriptsize
\tabcolsep=3pt
\begin{tabular}{|l|l|r|r|r|r|}
\hline
\multicolumn{1}{|c|}{Subject} & \multicolumn{1}{|c|}{URL} & \multicolumn{1}{|c|}{Entities} & \multicolumn{1}{|c|}{Operations} & \multicolumn{1}{|c|}{Rules} & \multicolumn{1}{|c|}{Rule parts} \\
\hline \hline
Cebu-pacific & www.cebupacificair.com 		& 8  & 10 &  	 & 31 \\
jBilling 		 & www.jbilling.com 					& 10 & 10 &   	 & 26 \\
App 				 & \multicolumn{1}{|c|}{---}	& 12 & 13 &     & 20 \\
\hline \hline
\textbf{Total} & 													& 	 & 33 &     & 77 \\
\hline
\end{tabular}
}
\label{tab:subjects}
\end{table}


\subsection{Experimental Setup}

\subsubsection{Implementation}
\label{sec:impl}

We implemented \tool{} as an Eclipse plugin, where users can model business
rules in a syntax-directed editor with various features such as auto completion
and on-the-fly detection of syntax errors. \tool{} uses \choco{} constraint
solver~\cite{Choco} for checking the satisfiability of sequences.
Currently, \tool{} provides a limited support for set types, where the number
of elements in sets is limited to size two. Also, since \choco{} does not 
provide functionality for extracting unsatisfied core, we use a heuristic-based
implementation to extract the core. While checking whether a concrete sequence
is satisfiable, \tool{} starts with the last operation of the sequence and
performs backward analysis by handling each operation and their rule parts. Therefore, when
a composed formula is unsatisfiable, it can happen only due to some
constraint in the most recently analyzed rule part. If the rule part includes
multiple constraints, \tool{} further analyzes each individual constraint to identify
the exact constraint that resulted unsatisfiability. Next \tool{} uses
variables involved in the identified constraint to extract the complete
unsatisfied core. Note that \tool{} may not extract 
the minimal unsatisfied core, however, the extracted core is sufficient
to suggest candidate operations. In future work, we plan to improve the efficiency
of our implementation by exploring advanced algorithms for extracting unsatisfied core~\cite{Liffiton:2008:ACM}
and also leverage other constraint solvers~\cite{DeMoura:2008}. We
also plan to leverage Green~\cite{VisserGD12} to cache the results of constraints
for further increasing the performance of our tool.

To show the efficiency of \tool{}, we 
implemented another tool, called \exhaust{}, that systematically explores sequences
without any guidance. Similar to \tool{}, \exhaust{} also uses a queue $q$
to store sequences. Given an operation ${\cal O}$ and a rule target $r$, 
\exhaust{} initially creates a sequence $seq$ with ${\cal O}\ [r]$ and adds $seq$
to $q$. For each $seq$, \exhaust{} associates a list of entities, referred to as \subject{ilist}. 
Initially, \subject{ilist} includes input entities of ${\cal O}$.
Next, \exhaust{} dequeues a sequence $qseq$ from $q$, removes an entity from its \subject{ilist},
and identifies all operations \{${\cal O}_1, {\cal O}_2, \ldots, {\cal O}_n$\}
that either create or modify that entity. \exhaust{} creates new sequences 
by adding each ${\cal O}_i$ to $qseq$. Also, if an ${\cal O}_i$ requires additional
input entities, it adds those entities to the \subject{ilist} corresponding to its newly
created sequence. Next, it adds all these newly created sequences to $q$
and repeats the process. Whenever \exhaust{} encounters a concrete sequence,
\ie{} sequence whose \subject{ilist} is empty, it uses \choco{} to check whether
the sequence is satisfiable. If yes, it returns the sequence; otherwise, continues
analyzing other sequences in the queue. \exhaust{} repeats this process until 
it finds a satisfiable sequence or it explores user-defined number of concrete
sequences. Being an unguided technique, \exhaust{} helps evaluate
the effectiveness of leveraging unsatisfied core as a guidance for searching
the desired sequence.

\subsubsection{Subjects and Rules}

We used two open-source applications and one IBM proprietary application, listed
in Table~\ref{tab:subjects}, as experimental subjects. Due to confidentiality reasons,
we refer to the proprietary application as \subject{App}. \subject{Cebu-pacific} is
an airlines application, \subject{jBilling} is an enterprise billing application,
and \subject{App} is a telecom application. Column~2 shows URL of each subject.
Columns~3--6 show additional details such as number of entities, operations,
and rules. For each subject, we identified a module that is likely to have large
number of interesting rules with respect to test data, and modeled those modules using our tool. In particular,
we used \textit{ticket cancellation} and \textit{generate invoice} modules for \subject{Cebu-pacific} and 
\subject{jBilling}, respectively. Overall, we modeled $33$ operations with $XX$ rules and
$77$ rule parts.\footnote{The complete dataset including the models and the generated
test sequences is available at \url{http://tinyurl.com/k4b4j8x} (also available on request
from the authors).}

\subsubsection{Method}

We applied both \tool{} and \exhaust{} on models created for each subject and generated
test sequences. For each rule part, we let each tool explore up to a maximum of $100$
concrete sequences. Next, we inspected generated sequences to ensure that 
they cover respective rule parts. To study the efficiency of both the tools, we measured the number of sequences
explored and also measured lengths of sequences produced by each tool.
All experiments were conducted on an Intel Core 2 Duo CPU
machine with 2.53 GHz and 8GB RAM. Next, we present the results of the studies.

\subsection{Coverage of Business Rules}

\begin{figure}[t]
\centering
\includegraphics[width=0.7\columnwidth, clip, trim = 18mm 120mm 140mm
  18mm]{figs/Study-1.pdf}
\caption{Effectiveness of the two techniques in covering business rules.}
\label{fig:effectiveness}
\end{figure}

Figure~\ref{fig:effectiveness} shows the results for all three subjects. 
In the figure, x axis shows subjects, whereas y axis shows percentages of rule parts
covered by each tool. For example, for \subject{Cebu-pacific},
\tool{} generated desired sequences for all $31$ rule parts, whereas \exhaust{} was able
to generate sequences only for $16$ rule parts. Overall, the results show 
that \tool{} handled $99$\% of rule parts, whereas \exhaust{}
handled only $74$\% of rule parts.

While analyzing our results, we identified that the rule parts that were 
not covered by \exhaust{} require complex sequences. In particular, \exhaust{}
could handle subjects such as \subject{jBilling}, where there are only a few
operations that create or modify each entity. On the other hand, it could not
handle subjects such as \subject{Cebu-pacific}, where there are several operations
that modify the same entity. Both \tool{} and \exhaust{} could not cover 
one rule part in \subject{jBilling} due to an issue with \choco{} solver. \choco{} produced
out of memory error while solving the composed formula for that rule part. We next present a complex sequence
generated by \tool{} for a rule part in \subject{Cebu-pacific}. This rule part is
related to refunding money to the passenger because of cancellation of flight due
to a delay of more than two hours.

{\scriptsize
\begin{alltt}
 int fund = 200, passenger = 1;
 CreateFund(fund, passenger, out: Fund fund);
 CreateTicket(passenger, out: Ticket ticket);
 int flight = 901, price = 100, departure = 10; 
 MakeSector(flight, price, out: Sector sector);
 AddSector(ticket, sector, fund,  
                out: Ticket ticket1, out: Fund fund1);
 int delay = 3, sectorid = 1; 
 DelayFlight(ticket1, sectorid, delay, out: Ticket ticket2);
 PartialCancellation(ticket2, sectorid, out: Ticket ticket3);
 Refund(ticket3, sectorid, fund1, 
                out: Ticket ticket4, out: Fund fund2);
\end{alltt}
}

\subject{Cebu-pacific} allows booking a ticket as multiple sectors, 
where each sector represents part of the journey from one city to another city. The airlines
has a refund policy where if the flight corresponding to any sector gets canceled
due to a delay of more than two hours, passengers can get refund so as to make alternate
travel arrangements. To cover this rule part that belongs to the operation
\subject{Refund}, a specific instance of \subject{Ticket} entity
is required. First, the ticket should include at least one sector and the passenger
should have sufficient funds to add a sector to the ticket. Next, the flight corresponding
to that sector should be delayed by more than two hours, and hence should be canceled
due to significant delay. \tool{} was able to successfully generate the preceding sequence,
whereas \exhaust{} was not able to generate any sequence for this rule part.
Overall, our results show promising benefits of our technique in effectively generating
complex sequences for covering business rules.

\begin{table}[t]
\caption{Efficiency of \tool{} and \exhaust{}.}
\centering
{\scriptsize
\tabcolsep=3pt
\begin{tabular}{|l|r|r|r|r|r|r|r|r|}
\hline
& \multicolumn{4}{|c|}{Sequence Length} & \multicolumn{4}{|c|}{\# of Sequences Explored} \\
\cline{2-9}
& \multicolumn{2}{|c|}{\tool{}} & \multicolumn{2}{|c|}{\exhaust{}} & \multicolumn{2}{|c|}{\tool{}} & \multicolumn{2}{|c|}{\exhaust{}}  \\
\cline{2-9}
\multicolumn{1}{|c|}{Subject} & \multicolumn{1}{|c|}{Max} & \multicolumn{1}{|c|}{Avg} & \multicolumn{1}{|c|}{Max} & \multicolumn{1}{|c|}{Avg} & \multicolumn{1}{|c|}{Max} & \multicolumn{1}{|c|}{Avg} & \multicolumn{1}{|c|}{Max} & \multicolumn{1}{|c|}{Avg} \\
\hline \hline
Cebu-pacific 	 &  7		& 5 &  9 &  5	 &  27 &  4	&  100 & 73 \\
jBilling		 	 &  6		& 3 &  6 &  3	 &  48 &  2	&  39  &  2 \\
App					 	 &  9		& 6 & 10 &  6	 &  51 &  5	& 100  & 46 \\
\hline
\end{tabular}
}
\label{tab:stats}
\end{table}

\subsection{Efficiency}

Table~\ref{tab:stats} shows data about lengths of sequences and number 
of sequences generated by both the tools. Columns~2--5 show the maximum 
and average lengths of sequences generated for all rule parts. 
On the other hand, Columns~6--9 show the number of 
sequences explored by each technique for all rule parts. 

Our results show that \tool{} generated relatively shorter sequences
compared to \exhaust{}. Furthermore, on average, \tool{} explored
only a few sequences compared to \exhaust{}. For example,
for \subject{Cebu-pacific}, \tool{} explored an average of $4$ sequences
(maximum of $27$), whereas \exhaust{} explored an average of $73$ sequences
for each rule part. Overall, the results show that \tool{} is highly efficient
compared to \exhaust{}.

% !TEX root = paper.tex
\section{Related Work}

At first blush, this problem seems to be reminiscent of the problem of
generating tests for programs written as control-flow graphs, with the
goal of exercising each acyclic path in the program, if possible.
There have been a number of techniques in the literature for test
generation. Mostly notably, techniques such as \textit{concolic}
testing attempt to identify a series of test data that would force
program execution thorough different paths~\cite{dart, concolic}.  Other
approaches are based on model checking, with the goal of creating test
inputs to reach specific program states.

However, the problem of test generation from business rules is
different. Business rules do not describe the implementation of a
system: rather they only describe a model and many of the concerns
that arise when dealing with control flow graph derived from real code
are not pertinent.

\subsection{Model Based Testing}

The set of business rules can viewed as a model of an actual system
that supports them and our approach then becomes a variant of model
based test generation~\cite{utting2012}.

Typical modeling languages used by model based testing systems include
UML sequence diagrams~\cite{nayak2009}, modeling specific languages
such as the Systems Modeling Language~\cite{friedenthal2011} and
finite state machine notations such as UML state
charts~\cite{offhut99}. While the business rules as presented in this
paper could be expressed in any of these notations it would be
cumbersome and require some degree of non-trivial encoding on part of
the author, making it unattractive to a non-programmer. Our business
rule language is designed so that rules are written in much the same
way as one would write them in prose thereby making it more accessible
to non-programmers.

Sriganesh and Ramanathan~\cite{sriganesh2012} extract business rules
from systems described in the Business Process Model and Notation
language, a graphical notation used to describe business
processes. The rule output is in a format similar to our formalism and
our algorithm could conceivably be applied to generate test sequences
from these rules.

\subsection{AI Planning}

Previous work has applied AI planning to software
testing~\cite{Scheetz99ai,Howe97testcase}. The sequencing of
operations done by our algorithm is similar to the AI planning
problem~\cite{Weld94} where actions with pre- and postconditions are sequenced
by a planner algorithm using either forward or backward chaining. The
algorithms used by AI planners are similiar the naive technique
presented in Section~\ref{sec:eval} which proved insuffecient for our
benchmarks. 

Part of an AI planning problem is the initial state of the program,
including what objects exists and which conditions that can be
assumed. Our technique does not need this: The types of objects that
exist in the system are defined in the model along with operations
that create them. A test sequence will construct any object needed to
cover a given rule part.

Paradkar et al~\cite{conf/icws/ParadkarSWJOSL07}.\ presents a system
for testing web services specified in the semantic markup language
OWL-S. The system is backed by an AI planner. OWL-S represents
operations in a manner similar to our language with pre- and
postcondtions. However each operation has only one associated pre- and
postcondition pair making it more akin to a rule in our language. If
one where to translate a model in our business rule language into
OWL-S each rule would be translated into a standalone operation. Such
a translation could potentially increase the size of the search space
and lead to spurious sequences being constructed. The focus of the
technique presented in the paper is on conformance testing, that is,
testing if a given implementation conforms to the specification. In
our approach the business rules are the specification and the aim is
instead to generates tests that cover all rules. The paper does not
mention the size of the models the technique was evaluated on nor the
size of the generated test sequences.

\subsection{Method Sequence Generation}

Do we need this section?

When testing object oriented code acheiving high branch coverage of a
method requires driving the reciever object and any objects passed as
arguments into a desired state. One apprach is to produce the desired
object state by generating sequences of method calls that create and
mutate objects similar to how we sequence
operations~\cite{pacheco2007,tillmann2008,thummalapenta2011}. However
the techniques are usually either based on random exploration or
symbolic execution neither of which apply in our business rule setting.

Something about unsat core

\section{Conclusion}

In this paper, we presented a new domain specific language for
modeling business rules that can be used to capture functional
specifications of enterprise systems. Previously such rules would only
be expressed in prose. We also defined four well-formedness properties,
which can also be verified mechanically. To enable non-programmers 
such as business analysts to create models we
developed an Eclipse IDE prototype, where a user can create and refine
models in a guided fashion. The tool includes a semantic checker that
can verify well-formedness of a model.

We presented an algorithm that mechanically generates test sequences to
exercise rules in the model. The algorithm translates the rules in the
model into logical expressions and uses a constraint solver to infer
the needed input values. For optimization, the algorithm uses a novel
approach to prune the search space based on unsatisfiable cores. 

Our technique was evaluated using three models written in the
business rule language. These models were derived from business rules
written in prose. The results show that our algorithm, with the
optimization, is able to cover 99\% of all business rules. Furthermore,
the results also show that our optimization is a crucial component in
achieving full coverage.

\bibliographystyle{abbrv}
\bibliography{paper}
\end{document}
