
\newcommand{\term}{\textit}
\newcommand{\lit}{\texttt}

\section{Rule Modeling and Checking}
\label{sec:model}

In this section, we discuss the notation for modeling business rules and the
static checking for completeness and consistency performed on formally captured
rules.

\begin{figure}[t]
\centering
{\small
\tabcolsep=3pt
\begin{tabular}{lll}
\term{RuleSpec} & ::= & \term{Entities} \term{Operations} \term{Triggers} \\
\\
\term{Entities} & ::= & \term{Entity} \term{Entities} | $\epsilon$ \\
\term{Entity} & ::= & \term{Enum} | \term{Object} \\
\term{Enum} & ::= & \lit{enum} \term{ID} \{ \term{EnumVals} \} \\
\term{EnumVals} & ::= & \term{ID} \term{EnumVals} | $\epsilon$ \\
\term{Object} & ::= & \lit{object} \term{ID} \{ \term{VarDecl} \} \\
\\
\term{Operations} & ::= & \term{Operation} \term{Operations} | $\epsilon$ \\
\term{Operation} & ::= & \lit{operation} \term{ID} \{ \term{Input}
\term{Creates} \term{Modifies} \term{Rules} \term{Next} \} \\
\term{Input} & ::= & \lit{input} : \term{VarDecl} \\
\term{Creates} & ::= & \lit{creates} : \term{VarDecl} \\
\term{Modifies} & ::= & \lit{modifies} : \term{VarDecl} \\
\term{Rules} & ::= & \term{Rule} \term{Rules} | $\epsilon$ \\
\term{Rule} & ::= & \lit{group} \term{ID} \{ \term{RuleParts} \} \\
\term{RuleParts} & ::= & \term{RulePart} \term{RuleParts} | $\epsilon$ \\
\term{RulePart} & ::= & \lit{rule} \term{ID} \{ \lit{pre} : \term{Expr} ;
\lit{post} : \term{Expr} \} \\
\term{Next} & ::= & \lit{next} : \term{ID} | $\epsilon$ \\
\\
\term{Triggers} & ::= & \term{ID} $\rightarrow$ \term{ID} \term{Triggers} | $\epsilon$ \\
\\
\term{VarDecl} & ::= & \term{TypeName} : \term{ID} \term{VarDecl} | $\epsilon$
\\
\term{TypeName} & ::= & \lit{int} | \lit{bool} | \lit{float} | \lit{string} |
\lit{set<\term{\textrm{TypeName}}>} | \term{ID} \\
\term{Expr} & ::= & \\
\end{tabular}
}
\caption{Partial rule-modeling syntax.}
\label{fig:model-syntax}
\end{figure}

\subsection{Rule-Modeling Language}

Overall, our approach models rules in the context of operations in the system
under test.  An operation has a set of input entities, a set of created
entities, a set of modified entities, and a set of rules, where each rule
consists of a set of precondition-postcondition pairs. For example,
Figure~\ref{fig:invoice} shows the operation for computing invoice totals in the
\subject{jBilling} application, which takes as input an invoice and modifies
certain attributes. The rules associated with this operation, which govern how
the invoice total and the customer's credit limit are updated, are modeled with
the operation in the form of precondition-postcondition pairs.

Figure~\ref{fig:model-syntax} presents the formal syntax of the rule-modeling
language (for clarity, we omit some of the details and present only the
important parts of the language). A \textit{rule specification} consists of
entities and operations. An \textit{entity} can be an object in the system (\eg
invoice, order, customer) or an enumerated type (\eg a customer's balance type
can be \subject{None}, \subject{Credit}, or \subject{Prepaid}).

The key part of the syntax, which models rules, is based on
operations. Formally, an \textit{operation} ${\cal O}$ is the tuple $(I, C, M,
R, o_n)$, where $I$ is the set of input entities read during the execution of
${\cal O}$, $C$ is the set of entities created by ${\cal O}$, $M$ is the set of
entities whose attributes are modified by ${\cal O}$, $R$ is the set of rules
that describe the behavior of ${\cal O}$, and $o_n$ is a succeeding operation
that, if specified, is the only operation that can execute after ${\cal O}$.

The notion of a \textit{succeeding operation} can simplify the modeling of
``coarse grained'' operations that have many rules associated with them. Such an
operation can be broken down into a sequence of finer-grained operations---with
the sequence specified via the \subject{next} clause---that have simpler
rules. The specification lets the test-generation algorithm (Section ??) ensure
that the atomic nature of the coarse-grained operation is preserved in the
finer-grained sequence: while chaining operations, the algorithm avoids
interleaving other operations in the middle of a finer-grained operation
sequence.  Moreover, such decomposition is necessary when the rules of an
operation have data dependences, which define an implicit ordering among the
rules. One of the well-formedness properties that we impose on rule
specifications (discussed in Section~\ref{sec:checking}) is data independence
among the rules of an operation.

A \textit{rule} $R = \{r_1, r_2, \ldots, r_k\}, k \geq 1,$ consists of a set of
rule parts. A \textit{rule part} $r$ is a precondition-postcondition pair, $p
\Longrightarrow q$, where $p$ and $q$ are boolean formulas such that if $p$
holds in the state before the operation, $q$ is true in the state resulting from
the execution of the operation. If the precondition of a rule part is true, we
say that the rule is \textit{applicable}.

Consider the rule illustrated in Figure~\ref{fig:invoice} for the
\subject{Compute Invoice Total} operation. The rule has three rule parts, each
of which consists of a precondition and a postcondition. The first rule part
pertains to the case where the customer's balance type is \subject{None}; the
second rule part is for the case where the balance type is \subject{Credit} and
the credit limit exceeds or equals the order total; the third rule part covers
the case where the balance type is \subject{Credit} and the order total exceeds
the credit limit.

Often in enterprise systems, the execution of an operation or a transaction
automatically triggers other transactions or operations. For example, in
\subject{JBilling}, customers with balance type \subject{Prepaid} have the
option of having their prepaid limit automatically recharged if it falls below a
threshold amount. Our rule-modeling syntax accommodates this via the
\subject{Triggers} clause, using which pairs of operations---where the first
operation automatically triggers the second operation---can be specified.

The primitive types (in addition to user-defined types or entities) include
integer, boolean, and string types. The model also accommodates sets of these
types.\footnote{\small In the currently implemented system, set types are
  handled in a restricted manner: by limiting the set size to two and specifying
  the predicates over a set in terms of its (two) elements. Section ??
  discusses the implementation details.}

\subsection{Rule Checking}
\label{sec:checking}
 
We define a few well-formedness properties on rule specifications to ensure
consistency and completeness, and that also facilitate operation chaining for
test generation. These properties are amenable to automated static
checking. Thus, we envision that the rules can be iteratively refined in a rule
editor, based on automatic (semantic) checking for property violations (in
addition to syntactic checking for conformance to the modeling syntax).

\paragraph*{Property 1: Rule-part Disjointedness}
In our notation, a rule part is intended to represent disjoint preconditions so
that when a rule is applicable, the precondition of only one rule part is true;
consequently, there is no ambiguity in identifying the relevant rule part for an
applicable rule. Formally, we define this property as follows. Let $R= \{r_1,
r_2, \ldots, r_k\}$ be a rule such that $k \geq 2$. Then, for all $r_i, r_j \in
R$ where $ r_i \coloneqq (p_i \Longrightarrow q_i)$ and $r_j \coloneqq (p_j
\Longrightarrow q_j)$, $(p_i \wedge p_j)$ must not be satisfiable. A simple
example of a rule specification that violates this property is $p_i = (a > 0)$
and $p_j = (a < 10)$; this specification represents ambiguous behavior when, for
example, $a = 5$.  The rule parts illustrated in Figure~\ref{fig:invoice} have
disjoint preconditions.

To check that a rule satisfies this property, first, we enumerate all pairs of
rule parts for the rule. Then, for each pair of rule parts (with preconditions
$p_i$ and $p_j$), we determine whether the boolean formula $(p_i \wedge p_j)$
has a solution; if it does, we flag a violation of the property.

\paragraph*{Property 2: Rule-part Completeness}
This property is intended to ensure that a rule specifies the complete operation
behavior for the variables mentioned in the rule. Let $R= \{r_1, r_2, \ldots,
r_k\}$ be a rule such that $k \geq 2$ and $r_i \coloneqq (p_i \Longrightarrow
q_i)$. Then, $\neg(p_1 \vee p_2 \vee \ldots \vee p_k)$ must not be
satisfiable. For example, the rule consisting of two rule parts with
preconditions $(a < 5)$ and $(a > 10)$, respectively, violates the completeness
property because the operation behavior for $5 \leq a \leq 10$ is left
unspecified.

To verify this property, our technique checks, for each rule that has two or
more parts, whether the formula $\neg(p_1 \vee p_2 \vee \ldots \vee p_k)$ has a
solution.

\paragraph*{Property 3: Rule Compatibility}
The rule compatibility property requires that for any set of applicable rules of
an operation, the postconditions of their relevant rule parts must not be
conflicting. To illustrate, consider rules $R_1 = \{r_1\}$ and $R_2 = \{r_2\}$
for an operation, such that $r_1 \coloneqq ((a > 0) \Longrightarrow (total =
10))$, $r_2 \coloneqq ((b > 0) \Longrightarrow (total = 20))$, and the two
preconditions are not disjoint (\ie $(a > 0) \wedge (b > 0)$ is
satisfiable). This pair of rules violates the compatibility property because the
postconditions are conflicting, whereas the corresponding preconditions can be
true simultaneously---note that the first precondition doesn't constrain the
value of \subject{b}, whereas the second precondition doesn't constrain the
value of \subject{a}. Thus, when both preconditions are true, it is not clear
what value would \subject{total} have after the operation.

To state this property formally, let $r_1 \coloneqq (p_1 \Longrightarrow q_1)$
and $r_2 \coloneqq (p_2 \Longrightarrow q_2)$ be the relevant rule parts of two
applicable rules of an operation. Then, if $(p_1 \wedge p_2)$ is satisfiable,
$(q_1 \wedge q_2)$ must be satisfiable. To verify this, our technique enumerates
all pairs of rules for an operation. Then, for each pair $R_1$ and $R_2$, the
techniques lists each combination $(p_1, p_2)$ of the preconditions of $R_1$ and
$R_2$, and checks the satisfiability of $(p_1 \wedge p_2)$ and the conjunction
of the corresponding postconditions.

If two rules violate the compatibility property, it might in fact indicate that
their rule parts can be merged into one rule. In the preceding example, $R_1$
and $R_2$ could be merged into one rule with two rule parts $R_m = \{r_{m_1},
r_{m_2}\}$, where $r_{m_1} \coloneqq (a > 0) \Longrightarrow (total = 10)$ and
$r_{m_2} \coloneqq (a \leq 0 \wedge b > 0) \Longrightarrow (total = 20)$. Note
adding the conjunct $a \leq 0$ (the negation of the precondition of $r_1$) to
the precondition of $r_{m_2}$ makes the two preconditions disjoint, which
ensures that $R_m$ satisfies the rule-part disjointedness
property. Alternatively, the negation of the precondition of $r_2$ could be
added to the precondition of $r_{m_1}$ to satisfy this property.

\paragraph*{Property 4: Rule Independence}
Finally, we enforce the restriction that there can be no data dependence between
the postcondition of one rule and the precondition of another rule of the same
operation. This property ensures that there is no implicit ordering among the
rules of an operation. When such an ordering exists between two rules of an
operation, the operation should be split into two operations in a sequence
specified using the \subject{next} clause.

We formalize this property as follows. Let ${\cal R} = \{R_1, R_2, \ldots,
R_n\}$ be the rules associated with an operation. For any $R_i, R_j \in {\cal
  R}$, let $V_{i, \mathit{post}}$ be the set of variables used in the
postconditions of the rule parts of $R_i$ and $V_{j, \mathit{pre}}$ be the set
of variables used in the preconditions of the rule parts of $R_j$. Then, $V_{i,
  \mathit{post}} \cap V_{j, \mathit{pre}} = \emptyset$. The checking of this
property is straight forward: for each pair of rules for an operation, the
technique computes the sets of variables used in the preconditions of one rule
and the postconditions of the other rule, and verifies that the two sets are
non-intersecting.
