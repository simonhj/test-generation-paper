
\newcommand{\term}{\textit}
\newcommand{\lit}{\texttt}

\section{Business Rule Modeling and Checking}
\label{sec:model}

In this section, we discuss the notation for modeling business rules and the
static checking for completeness and consistency performed on formally captured
rules.

\begin{figure}[t]
\centering
{\small
\tabcolsep=3pt
\begin{tabular}{lll}
\term{RuleSpec} & ::= & \term{Entities} \term{Operations} \\
\\
\term{Entities} & ::= & \term{Entity} \term{Entities} | $\epsilon$ \\
\term{Entity} & ::= & \term{Enum} | \term{Object} \\
\term{Enum} & ::= & \lit{enum} \term{ID} \{ \term{EnumVals} \} \\
\term{EnumVals} & ::= & \term{ID} \term{EnumVals} | $\epsilon$ \\
\term{Object} & ::= & \lit{object} \term{ID} \{ \term{VarDecl} \} \\
\\
\term{Operations} & ::= & \term{Operation} \term{Operations} | $\epsilon$ \\
\term{Operation} & ::= & \lit{operation} \term{ID} \{ \term{Input}
\term{Creates} \term{Modifies} \term{Rules} \term{Triggers}\} \\
\term{Input} & ::= & \lit{input} : \term{VarDecl} \\
\term{Creates} & ::= & \lit{creates} : \term{VarDecl} \\
\term{Modifies} & ::= & \lit{modifies} : \term{VarDecl} \\
\term{Rules} & ::= & \term{Rule} \term{Rules} | $\epsilon$ \\
\term{Rule} & ::= & \lit{group} \term{ID} \{ \term{RuleParts} \} \\
\term{RuleParts} & ::= & \term{RulePart} \term{RuleParts} | $\epsilon$ \\
\term{RulePart} & ::= & \lit{rule} \term{ID} \{ \lit{pre} : \term{Expr} ;
\lit{post} : \term{Expr} \} \\
\term{Triggers} & ::= & \lit{next} : \term{ID} | $\epsilon$ \\
\\
\term{VarDecl} & ::= & \term{TypeName} : \term{ID} \term{VarDecl} | $\epsilon$
\\
\term{TypeName} & ::= & \lit{int} | \lit{bool} | \lit{float} | \lit{string} |
\lit{set<\term{\textrm{TypeName}}>} | \term{ID} \\
\term{Expr} & ::= & \\
\end{tabular}
}
\caption{Partial rule-modeling syntax.}
\label{fig:model-syntax}
\end{figure}

\subsection{Rule-Modeling Language}

Overall, our approach models rules in the context of operations in the system
under test.  An operation has a set of input entities, a set of created
entities, a set of modified entities, and a set of rules, where each rule
consists of a set of precondition-postcondition pairs. For example,
Figure~\ref{fig:invoice} shows the operation for computing invoice totals in the
\subject{JBilling} application, which takes as input an invoice and modifies
certain attributes. The rules associated with this operation, which govern how
the invoice total and the customer's credit limit are updated, are modeled with
the operation in the form of precondition-postcondition pairs.

Figure~\ref{fig:model-syntax} presents the formal syntax of the rule-modeling
language (for clarity, we omit some of the details and present only the
important parts of the language). A \textit{rule specification} consists of
entities and operations. An \textit{entity} can be an object in the system (\eg
invoice, order, customer) or an enumerated type (\eg a customer's balance type
can be none, credit, or prepaid).

The key part of the syntax, which models rules, is based on operations. An
\textit{operation} has a set of rules associated with it, and each rule is
composed of a set of rule parts. A \textit{rule part} consists of set of
precondition-postcondition pairs, $p \Longrightarrow q$, where $p$ and $q$ are
boolean formulas, and if $p$ holds in the state before the operation, $q$ is
true after the execution of the operation.

\subsection{Rule Checking}
